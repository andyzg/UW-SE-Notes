\chapter{Electric Fields}
	\section{Properties of an Electric Charge}
		\begin{itemize}
			\item Two types of charge: \textbf{Positive} and \textbf{Negative}
			\item Charges of the same sign repel one another and charges with opposite signs attract one another.
			\item Electric charge is always conserved.
		\end{itemize}
	\section{Charging Objects by Induction and Conduction}
		\subsection{Insulators and Conductors}
		\begin{itemize}
			\item Conductors are a material where electrons not bound to atoms and are free to move through the material.
			\item Insulators are materials in which all electrons are bound to atoms and cannot move freely through the material.
		\end{itemize}
		\subsection{Charging by Induction}
			Using a charged object, charges in a conductor can be repulsed out of an object through ground. The objects do not have to touch for this to happen.
		\subsection{Charging by Conduction}
			Rub two objects so that charges on one rub off to the other.
	\section{Coulomb's Law}
		To generalize the properties of the electric force, we find that the magnitude of the electric force between two point charges is given by Coulomb's Law:\\
		\centerline{$F_e = k_e \frac{|q_1||q_2|}{r^2}$}\\
		where $k_e$ is known as Coulomb's constant and has the value:\\
		\centerline{$k_e = 8.9876 \times 10^9 N \cdot m^2/C^2 = \frac{1}{4\pi \epsilon_0}$ }\\
		Note that the force is a vector and with multiple charges, the force is equal to the sum of the vector forces. As mentioned previously, the direction depends on the two charges.
	\section{Particle in an Electric Field}
		An electric field is said to exist in the region of space around a charged object, the source charge. We define the electric field vector $\vec{E}$ at a point to be:\\
		\centerline{$\vec{E} \equiv \frac{\vec{F}_e}{q_0} = k_e \frac{q}{r^2}\hat{r}$}\\
		Like the force, an electric field at a certain point is the sum of all electric fields due to multiple point charges.
	\section{Electric Field of a Continuous Charge Distribution}
		When calculating the electric field at a point due to a continuous distribution of charge such as a surface, we can divide the charge distribution into many small charge $\Delta q$.\\
		We then take the sum of all of these charges:\\
		\centerline{$\vec{E} = k_e \int_{}^{} \frac{dq}{r^2} \hat{r}$} \\
		Popular examples would include a uniform ring of charge where many of the forces would cancel out due to symmetry and you'd calculate the formula with a $\cos \theta$ to only get 1 component. This approach would be the same for a disk of uniform charge.
	\section{Electric Field Lines}
		The electric field vector $\vec{E}$ is tangent to the electric field line at each point. The denser the number of electric field lines, the stronger the field and vice versa.\\
		For \textbf{positive charges} the field lines are directed radially outward.\\
		For \textbf{negative charges} the field lines are directed radially inward.\\
		The number of lines drawn leaving a charge is proportional to the \textbf{magnitude} of the charge.
	\section{Motion of a Charged Particle in a Uniform Field}
		Since the force is denoted as $\vec{F}_e = q\vec{E} = m\vec{a}$, then we can isolate $\vec{a}$ and determine the motion of the particle based on the acceleration. If the electric field is uniform, then we know the particle is under constant acceleration.
