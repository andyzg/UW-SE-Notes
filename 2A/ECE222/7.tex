\section{Subroutines}
  \subsection{The Stack}
    \begin{itemize}
      \item LIFO
      \item Each thread/process has a call stack growing from high to low
      memory address
      \item Typical memory map (32 bit addressing)
      \item r13(alias sp) is the stack pointer
      \item Push[[r0]] onto the stack
      \begin{lstlisting}
        STR r0, [sp, #-4]! // \! == update/write, (pre-indexed)
      \end{lstlisting}
      \item Pop from stack into r0
      \begin{lstlisting}
        LDR r0, [sp], #4 // post indexed
      \end{lstlisting}
    \end{itemize}

  \subsection{Calling}
    \begin{itemize}
      \item Branch to subroutine instructions store return address in the link
      register r14 (alias lr)
      \item Branch and link: BL\{cond\} label (invocation)
      \begin{lstlisting}
        // pc <- [pc] + offset // offset in [-16MB, 16MB]
        r14 <- [return address] // pc of the next instruction
      \end{lstlisting}
      \item Branch, link and exchange: BLX {cond} Rn (invocation)
      \begin{lstlisting}
        eg: BLX r1
        // pc <- [r1], r14 <- [pc]
        // Greather range than BL [-2GB, 2GB]
      \end{lstlisting}
    \item Branch exchange: Bx {cond} Rn (return)
      \begin{lstlisting}
        e.g. Bx lr
        // pc <- [r14]
      \end{lstlisting}
    \end{itemize}

  \subsection{Load/Store Multiple}
    \begin{itemize}
      \item Push/pop multiple register values to/from stack
      \item Syntax \textless{} op\textgreater{} \{mode\}\{cond\} Rn\{!\},
      reglist
      \begin{itemize}
        \item[op] = LDM load muliple, or SDM store multiple
        \item[mode] = IA increment after, or DB decrement before
        \item[!] = writeback (update the stack pointer)
        \item[reglist] = comma separated list of regs or reg ranges
      \end{itemize}
      \item STMFD is a synonym for STMDB (push)
      \item LOMFD is a synonym for LDMIA
      \begin{itemize}
        \item[FD] = full descending stack
      \end{itemize}
      \begin{lstlisting}
        e.g.\ push r4, r5, r6 and r14 onto the stack
        STMFD sp!, [r4, r6, lr]
        r4  | 0000 0004
        r5  | 0000 0005
        46  | 0000 0006
        r13 | 1000 0200 <- sp
        r14 | 0000 0100 <- lr
        Memory after pushing
        1000 01F0 | 0000 0004 <- sp
        1000 01F4 | 0000 0005
        1000 01F8 | 0000 0006
        1000 01FC | 0000 0100
        1000 0200 | prev valu <- previous sp

        e.g.\ pop from stack into r4, r5, r6, pc
        LDMFD sp! {r4---r6, pc}
      \end{lstlisting}
    \end{itemize}

  \subsection{AAPCS (Arm Architecture Procedure Call Standard)}
    \begin{tabular}{c c c c}
      registers & synonyms & callee preserved & function\\
      r0-r3 & a1-a4   &   no  & argument/result/scratch regs\\
      r4-r1 & v1-v8   &   yes & local variable\\
      r12   & p       &   no  & intro procedure/scracthing\\
      r13   & sp      &   yes & stack pointer\\
      r14   & lr      &   no  & link register\\
      r15   & pc      &   no  & program counter
    \end{tabular}
    \paragraph{Guidelines}
    \begin{itemize}
      \item Preserve and restore v1 --- v8 (r4 --- r11) if you modify it
      \item Anything pushed in the stack must be popped/removed
      \item Return values are in r0 (and r1 --- r3 as needed)
      \item Pass parameters via registers first (faster)
      \item pass additional parameters via stack
    \end{itemize}
