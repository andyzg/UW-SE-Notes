  \section{Memory}
    \begin{itemize}
      \item A processor can access a finite amount of physical memory,
      determined by the \# of address pins
      \item Memory is measured in binary units but reported with S.I. prefixes
      (e.g. 1kB = 1024 bytes)
      \item Hard disks are measured in decimal units (e.g. 1kB = 1000 bytes)
      \item IEC introduced binary units to eliminate confusion (e.g. 1kiB = 1024
      bytes)
    \end{itemize}
    \paragraph{Endianness}
    \begin{itemize}
      \item Big endian: MSB (most sig.\ byte) is at low address, LSB at high
      address
      \item Little-endian: MSB at high address, LSB at low address
    \end{itemize}
    \paragraph{Ex:} \textbf{12}34AB\textbf{CO}
    \begin{center}
      \begin{tabular}{c c}
        Big Endian &  \\
        0 & 12 \\
        1 & 34 \\
        2 & AB \\
        3 & CO \\
      \end{tabular}
      \begin{tabular}{c c}
        Little Endian & \\
        0 & CO \\
        1 & AB \\
        2 & 34 \\
        3 & 12 \\
      \end{tabular}
    \end{center}
\chapter{ARM}
  \section{Background}
    \begin{itemize}
      \item Acon/Adoned RISC Machines
      \item License designs to other companies to manufacture
      \item Target low power/low cost
    \end{itemize}
    Documentation:
    \url{http://infocenter.arm.com/help/index.jsp}

  \section{Design Principles}
    RISC but with some CISC characteristics\\
    RISC\:
    \begin{itemize}
      \item Fixed instruction size
      \item load/store architecture
    \end{itemize}
    CISC\:
    \begin{itemize}
      \item Autoincrement/decrement addressing modes
      \item Move multiple values from registers to memory, in 1 instruction
      \item Condition codes
    \end{itemize}
  \section{Memory}
    \begin{itemize}
      \item Data sizes:
        \begin{itemize}
          \item Word = 32 bits
          \item Half word = 16 bits
          \item Byte = 8 bits
        \end{itemize}
      \item Word addresses are `word aligned' (multiple of 4)
      \item Little or big endian
      \item Loads of half words or bytes at  200 extended or sign extended to
        32 bits
    \end{itemize}

  \section{Registers}
    Registers\:
    \begin{itemize}
      \item All registers are 32 bits
      \item 13 General purpose registers\: R0-R12, and
        \begin{itemize}
          \item[R13] is the stack pointer (SP)
          \item[R14] is the link register (LR)
          \item[R15] is the program counter (PC)
        \end{itemize}
      \item Condition code flags\:
        \begin{itemize}
        \item[R28 (V)] Overflow
        \item[R29 (C)] Carry out
        \item[R30 (Z)] Zero
        \item[R31 (N)] Negative
        \end{itemize}
      \item Program status register
    \end{itemize}

  \section{Instruction Set}
    \subsection{Variations}
    3 variations
    \begin{itemize}
      \item ARM --- 32 bit
        Thumb --- 16 bit (compact, limited instructions, 8 operands)
        Thumb 2 --- Mix of 16 and 32 bit, Cortex M3 in lab
    \end{itemize}

    \subsection{Data Processing Instructions}
    Most have this format\\
    \begin{lstlisting}
      <op>{flags}{cond} Rd, Rn, Op2
    \end{lstlisting}
    \begin{itemize}
      \item[Op2] Operand 2 (right operand)
      \item[Rn] Source register (Left operand)
      \item[Rd] Destination register
      \item[\{cond\}] Execute if condition is true
      \item[\{flags\}] E.g.\ S =\textgreater{} set condition code flags
      \item[\textless{} op\textgreater] Operation meumonic
    \end{itemize}
    \paragraph{E.g.}
    \begin{lstlisting}
      ADDEQ R2, R0, #1
      if Z==1, R2 <- [R0] + 1
    \end{lstlisting}

    Operand 2
    \begin{itemize}
      \item an 8 bit constant (optionally rotated)
      \item Register value (Rm) optionally shifted
      \begin{itemize}
        \item[LSL] Logical Shift Left, shift all bits to the left with 0 as LSB
          (multiply)
        \item[LSR] Logical Shift Right, shift all bits to the right with 0 as
          MSB (unsigned divide by two)
        \item[ASR] Arithmetic Shfit Right, shift all bits to the right, MSB
          becomes itself (signed divide by two)
        \item[RDR] Rotate right
        \item[RRX] Rotate right extend
      \end{itemize}
    \end{itemize}
    \paragraph{E.g.}
    \begin{lstlisting}
      ADD r2, r0, r1, LSL #2
      r2 <- [r0] + [r1] << 2
      = r2 <- [r0] + 4*[r2]
    \end{lstlisting}

    \begin{itemize}
      \item[Arithmetic] ADD, ADC (add with carry), SUB, SBC, RSB (reverse
      subtract)
      \item[Logical] (bitwise) AND, ORR, EOR, BIC (and not, bitwise clear), ORN
    \end{itemize}
    \begin{lstlisting}
      (or not), (no NOT) --- EOR Rd, Rn \#0xffffff
    \end{lstlisting}

