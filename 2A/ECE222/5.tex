\section{Memory}
  \subsection{Definitions}
    \paragraph{DCD} = Declare Data
    \begin{lstlisting}
      DCD 5
      DCB 1, 2, 3, 4, 5
    \end{lstlisting}
    This is very similar to declaring
    \begin{lstlisting}
      const int N = 5
      const int ARRAY[] = [1, 2, 3, 4, 5]
    \end{lstlisting}

    \paragraph{ADR} = Put address in register
    \begin{lstlisting}
      int *ptr = ARRAY,
      int total = ptr[0];
      int a = ptr[1];
      total += a;
    \end{lstlisting}

  \paragraph{Example} ~\\
    pc relative: LDR\{size\}\{cond\} Rt, label
    \begin{itemize}
      \item label is translated into an offset from the instruction
      \item data is loaded from \textless address\textgreater = PC + 4 + offset
      \item offset $\in$ [[ pc, \#28 ]]
    \end{itemize}

  \paragraph{E.g.}
  \begin{lstlisting}
    LDR r1, DATA1 ; DATA 1 is 322 bytes after LDR
    =>
    LDR r1, [pc, #28] ; pc incremented by 4 during fetch
    // r1 <- [[pc] + 28]
  \end{lstlisting}

  load address into register: ADR\{cond\} Rd, label
  \paragraph{e.g.}
  \begin{lstlisting}
    ADR r1, DATA1
    // r1 <- [pc] + 28
  \end{lstlisting}

\section{Pseudo Instructions}
  \begin{itemize}
    \item don't match an existing machine instruction
    \item translated by the assembler into an appropriate instruction(s)
    \paragraph{e.g.}
    \begin{lstlisting}
      LDR {cond} Rt = <expr> ; label or numeric expression at end
    \end{lstlisting}
    \item converted into
    \begin{itemize}
      \item MOV or MVN, if possible, or
      \item adds the value to the literal pool (program contants, bottom of
      file) and generates a LDR instruction with pc + relative addressing from
      demo0.s file
      \begin{lstlisting}
        LDR r1 = SUM
      \end{lstlisting}
      \item couldn't use ADR because the distance from LDR (0x114) to sum (0x1000
      0000) exceeds $\pm 4095$
    \end{itemize}
    \item Stores 0x1000 0000 at address 0x134 and replaces it with:
    \begin{lstlisting}
      LDR r1, [pc, #28]
    \end{lstlisting}
  \end{itemize}

\section{Branch Instruction}
  \begin{itemize}
    \item Changes control flow by adding an offset to the PC
    \item Format: B\{cond\} label
    \begin{itemize}
      \item[cond] only execute if condition is true
      \item[label] assembler replaces with pc relative offset
    \end{itemize}
    \item condition code suffixes (SIGNED)
    \begin{itemize}
      \item[EQ] equal (to zero)
      \item[NE] not equal
      \item[GT] greater than
      \item[GE] greater or equal
      \item[LT] less than
      \item[LE] less or equal
      \item[PL] plus ($>= 0$) ignores overflow
      \item[MI] minus ($<0$) ignores overflow
    \end{itemize}
    \item condition code suffixes (UNSIGNED)
    \begin{itemize}
      \item[VS] overflow set
      \item[VC] overflow clear
      \item[Al] always (default)
    \end{itemize}

    \paragraph{E.g.} SUBS r2, r2, \#1 // r2 <- [r2] - 1
    \begin{lstlisting}
      BGT LOOP
      do { int a = *ptr;
           total += a;
           ptr++;
           counter--;
         } while (counter > 0);} // demoX.s
    \end{lstlisting}
  \end{itemize}

