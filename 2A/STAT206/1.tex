\section{Study Error versus Sample Error}
  \subsection{Classic Example of Study Error}
    Suppose you work for Rob Ford and you want to estimate the proportion of
    Torontonian voters who will vote for Rob Ford on election day.\\
    You run a telephone survey to estimate this proportion. What is the study
    error here?
    \paragraph{Study Error} is systematic skewing of results from the design of
    the setup

    \paragraph A: We have ruled out Toronto voters without phones

    \paragraph 2: Depending on when survey is carried out, response rate could
    be affected

    \paragraph 3: Voters who live outside Toronto can have Toronto cell \#'s

    \paragraph 4: Non response can be high. Responers and non responders might
    have different opinion of Rob Ford

    \paragraph{} These study errors cannot be addressed by the techniques of
    Stat 206. Instead, these errors must be addressed at the design stage

    \paragraph{} Sampling Error occurs when the chosen sample is not a good
    representation of the population for the desired attribute.

    \paragraph{Eg:} Population ${A, B, C}$, say they are 3 users of an Android
    app you have written. \\
    Variable time to complete somt task measured in seconds.\\

   Results =
    \begin{tabular}{l | l | l | l }
      User    & A & B  & C  \\
      \hline
      Time(s) & 2 & 10 & 18 \\
    \end{tabular}

    \begin{itemize}
      \item Taking unit B gives 10s - good!
      \item Taking unit A gives 2s - too low
      \item Taking unit C gives 18s - too high
    \end{itemize}
    This type of sampling error can be quantified using STAT206 techniques

