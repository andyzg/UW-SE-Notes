\chapter{Probability}
  \section{Definitions}
    \paragraph{Probability} measure the uncertainty associated with an event.
    An event is something that might occur
    \begin{itemize}
      \item Classical: $\frac{Number of ways event can occur}{Total number of
      equally likely outcomes}$
      \item Relative Frequency: Proportion of times the event occurs, as the
      number of trials approaches infinity
      \item Subjective: Estimates of probability that the event occurs, based
      on subjective opinion
    \end{itemize}

    \paragraph{Experiment} is a repeatable phenomenon or process
    \paragraph{Trial} is a single repetition of an experiment
    \paragraph{Sample Space}, S, is the set of distinct outcomes for an
    experiment or process
    \paragraph{Discrete} A sample space is discrete if it has a finite or
    countably infinite number of simple events. Otherwise it is non discrete or
    continuous

    \paragraph{Mutually Exclusive} means two events never occur simultaneously

    \paragraph{Complement} of an event and an event are always mutually
    exclusive

    \paragraph{Uniform distribution} The total probability is uniformly
    distributed among all possible outcomes

    \paragraph{Permutation} is the number of ways to arrange $r$ out of $n$
    objects: $ n ^ {(r)} = \frac{n!}{(n-r)!} = n(n-1)(n-2) ... (n-r+1)$

    \paragraph{Combinations} If we don't care about the order of objects, but
    just which objects are chosen, the number of ways to choose $r$ out of $n$
    items is $\binom{n}{r} = \frac{n!}{r!(n-r)!}$

    \paragraph{Set Operations}
    \begin{itemize}
      \item $AB$ or $A\cap B$ is the intersection of two
      events.
      \item$A \cup B$ is the union of two events.
      \item $\bar{A}$ is the complement of A, not event A
    \end{itemize}

    \paragraph{Conditional} The probability of event A, conditional on the
    occurence of event B, denoted by $P(A|B)$ is $P(A|B) = \frac{P(A \cap
    B)}{P(B)}$, $P(B) \neq 0$

    \paragraph{Independent} Two events are said to be independent iff $P(A \cap
    B) = P(A)P(B)$. This implies that $P(A|B) = P(A)$, $P(B|A) = P(B)$. In
    other words, events A and B are independent if whether B occurs does not
    influence whether A occurs, and vice versa

    \paragraph{Bayes' Theorem} Suppose A and B are any two events in S, then
    $P(B|A) = \frac{P(A|B)P(B)}{P(A|B)P(B) + P(A|\bar{B})P(\bar{B})}$
