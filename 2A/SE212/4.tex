\chapter{Module 4 - Theories}
  \section{Definitions}
    \paragraph{Enthymeme} is an argument that contains a hidden premise, that
    is, an argument that contains unstated premises that are obviously true

    \paragraph{Theory}
    \begin{itemize}
      \item A set of statements or principles devised to explain a group of
      facts or phenomena, especially one that has been repeatedly tested or is
      widely accepted and can be used to make predictions about natural
      phenomena
      \item A set of theorems that constitute a systematic view of a branch of
      mathematics
    \end{itemize}

    \paragraph{Leibnizs Law} if $t_1 == t_2$ is a theorem, then so is $P[ t_1 / x ]
    \Leftrightarrow P[ t_2 / x ]$\\
    Lebinizs Law is generally referred to as the ability to substitute equals
    for equals

    \paragraph{Normal interpretations} are interpretations in which the symbol
    == is interpreted as equality on the objects of the domain.

    \paragraph{Induction}
      \begin{itemize}
        \item $P(0)$ is called the base case
        \item $P(k)$ is called the induction hypothesis
        \item $\forall k \wedge P(k) \Rightarrow P(suc(k))$ is called the
        induction proof or inductive step
      \end{itemize}

    \paragraph{Deduction} is showing a conclusion follows frmo the stated
    premises using rules of inference

    \paragraph{Philosophical induction} is the process of deriving general
    principles from particular observations

    \paragraph{Recursive function} is one that is defined in terms of itself
    and certain terminating clauses

    \paragraph{Satisfiable Modulo Theories}
      is a decision problem for logical formulas with respect to combinations
      of background theories experessed in classical first order logic with
      equality.

    \paragraph{Rule of Thumb} when formalizing a set of sentences.
    \begin{enumerate}
      \item Identify all possible logical connectives. Recall the heuristic
        from propositional logic: pick the smallest phrases without and or if
        then etc.
      \item Identify what might be quantified, implicit quantifiers, constants
      \item Identify the possible predicates and functinos. The quantified
        variables must be arguments to the predicates
      \item Look for similar phrases to determine how much detail you need to
        provide
      \item Think of an outline of how your proof will proceed.
    \end{enumerate}

    \paragraph{Rule of Thumb} for types
    \begin{enumerate}
      \item If a unary predicate is used in all premises and conclusions as
        part of the antecedent of a universally quantified formula, it is a
        good candidate to make a type
      \item If a unary predicate is used in all premises as one conjuct of an
        existentially quantified formula, it is a good candidate to make a type
    \end{enumerate}
