\chapter{Module 2}
  \section{Definitions}
    \subsection{Elements of a logic}
      Logic consists of \textbf{syntax}, \textbf{semantics} and \textbf{proof
      theory}
    \subsection{Syntax}
      `well-formed formula' (wff), is a word that is a part of a formal
      language
    \subsection{Semantics}
      $\vdash$ is entails. \\
      Ex: `$\vdash p$' means the formula $p$ is valid, where $p$ is a wff in the
      logic.\\
      Ex: $p_1, p_2,\ldots,p_n \vdash q$ means from the premises ($p$'s), we
      may conclude $q$ where they're all wff.
    \subsection{Proof Theory}
      Define `$\models$' as proves. It's a way to calculate $p_1,
      p_2,\ldots,p_n \models q$, meaning there's a way to determine if $q$ is
      true if $p_1, p_2,\ldots,p_n$ are true\\
      There may be multiple proof theories indicated by a subscript e.g.
      $\models_{ND}$ for natural deduction proof theory.

      \paragraph{Proof Theory} are methods that manipulate strings of symbols
      base on pattern matching. There may be multiple ways to prove a formula.

      \paragraph{Sound} If $p_1, p_2,\ldots,p_n \models q$ (proof), then $p_1,
      p_2,\ldots,p_n \vdash q$ (valid)

      \paragraph{Complete} If $p_1, p_2,\ldots,p_n \vdash q$ (valid) then
      $p_1,p_2,\ldots,p_n \models q$ (proof)
  \section{Syntax}
    \subsection{Composition}
      A formula in propositional logic consists of constant symbols
      (\textbf{true} and \textbf{false}), proposition letters, propositional
      connectives ($\lnot, \land, \lor, \Rightarrow, \Leftrightarrow$), and
      brackets

    \subsection{Symbol Definitions}
      \begin{itemize}
        \item[$\lnot$] Negation
        \item[$\land$] And
        \item[$\lor$] Or
        \item[$\Rightarrow$] Implication
        \item[$\Leftrightarrow$] Equivalent
      \end{itemize}

    \subsection{Rules and Definitions}
      \begin{itemize}
        \item Brackets around the outermost formula are usually omitted.
          Priority is ($\lnot, \land, \lor, \Rightarrow, \Leftrightarrow$)
        \item All binary logical connectives are \textbf{right associative}
        \item $a\land b$ is \textbf{conjucts}, $a\lor b$ is \textbf{disjuncts}
        \item Contrapositive of $a\Rightarrow b$ is $\lnot b\Rightarrow \lnot
          a$
        \item Prime propositions are declarative sentences i.e.\ sentences that
          are true or false
        \item Propositional letters are $a, b, p, q$
        \item Prime compositions are indecomposable, compound composition are
          decomposable
        \item The connective `and' used in logic is commutative
        \item Watch out for the false implies everything problem

      \end{itemize}
    \section{Semantics}
      \subsection{Definition}
      \paragraph{Semantics} means meaning, providing an interpretation/functions of
        expressions in one world in terms of values in another world. Proof
        theories transform wff in ways that respect semantics.\\
        The syntax of the propositional logic is the \textbf{domain} of the
        semantic function whereas the set of truth values is the \textbf{range}

        \paragraph{Boolean Valuation} is a function $v$ from the set of
        formulas in propositional logic to the set $T_r$. Boolean valuation is
        also called a model or an interpretation.
        \begin{itemize}
          \item $v(false) = F, v(true) = T$
          \item $v(\lnot p) = NOT (v(p))$
          \item For the connectives:
          \begin{itemize}
            \item $v(p\land q) = v(p) AND v(q)$
            \item $v(p\lor q) = v(p) OR v(q)$
            \item $v(p\Rightarrow q) = v(p) IMP v(q)$
            \item $v(p\Leftrightarrow q) = v(p) IFF v(q)$
          \end{itemize}
        \end{itemize}

      \subsection{Semantics of Propositional Logic}
        \paragraph{NOT} takes a truth value and returns a truth value
        \paragraph{AND, OR, IMP, IFF} take two truth values and return a truth
        value that correspond to $\land, \lor, \Rightarrow, \Leftrightarrow$

        \paragraph{Truth Tables} have a row for each possible boolean valuation,
          a column for each subformula for the formula, and the formula itself,
          and each cell contains the truth value given by the boolean valuation
          of that row. We can use truth tables to determine if a formula is
          satisfiable/tautology/contradiction/contingent

        \paragraph{Satisfiability} If a formula is \textbf{satisfiable}, then
        there exists a Boolean valuation $v$ such that $v(p) = T$

        \paragraph{Tautologies} A propositional formula $p$ is a
        \textbf{tautology} or \textbf{valid} if $v(p) = T$ for all Boolean
        valuations $v$. When a formula $q$ is a tautology, we write
        \begin{center}
           $\vdash q$
        \end{center}

        \paragraph{Logical Implication} A formula $p$ \textbf{logically
        implies} a formula $q$ iff for all Boolean valuations $v$, if for all
        premises $v(p_i) =
        T$, then $v(q) = T$, meaning
        \begin{center}
          $p \vdash q$ which is equivalent to $\vdash p \Rightarrow q$
        \end{center}

        \paragraph{Contradiction} A propositional formula $a$ is a
        \textbf{contradiction} if $v(a) = F$ for all Boolean valuations $v$

        \paragraph{Contingent} A \textbf{contingent} is one that is neither a
        \textbf{tautology} nor a \textbf{contradiction}

        \paragraph{Logical Equivalence} Two formulas are \textbf{logically
        equivalent} iff their equivalence is a tautology i.e. $v(p) = v(q)$ for
        all $v$.
        \begin{center}
          $p\leftrightarrow q$ which also means $\vdash p \Leftrightarrow q$
        \end{center}
        \begin{itemize}
          \item[$\leftrightarrow$] Logical equivalence
          \item[$\Leftrightarrow$] Material equivalence
        \end{itemize}

        \paragraph{Consistency} A collection of formulas is \textbf{consistent}
        if there exists a boolean valuation where all the formulas can be true
        simultaneously. If a set of formulas in the antecedent of an
        implication, they can be used to prove a contradiction.

  \section{Proof Theories}
    \subsection{Use Case}
      Since truth tables grow exponentially, we can use a \textbf{proof theory}
      instead to determine whether a formula is a tautology. As long as the
      proof theory is \textbf{sound}, we can use proof theory in place of truth
      tables to determine tautologies (and valid arguments).

    \subsection{Transformational Proofs}
      \paragraph{Transformational Proofs} is a means of determining that two
      wff formulas of propositional logic $p$ and $q$ are logically equivalent
      by the repeated exchange of subformulas of $p$ for logically equivalent
      subformlas that result in $p$ being transformed into $q$. Each step must
      follow a logical law expressed by $\Lleftarrow\Rrightarrow$.\\
      $p\Lleftarrow^?\Rrightarrow q$ means `Show by transformational proof that
      $p\Lleftarrow\Rrightarrow q''$'
      \paragraph{Transformational Proof Rules}
      \begin{itemize}
        \item[Comm] $p\land q \Lleftarrow\Rrightarrow q\land p$
        \item[Lem] $p\lor\lnot \Lleftarrow\Rrightarrow true$
        \item[Contr] $p\land\lnot p\Lleftarrow\Rrightarrow false$
        \item[Impl] $p \Rightarrow q\Lleftarrow\Rrightarrow \lnot p \lor q$
        \item[Idemp] $p \land p\Lleftarrow\Rrightarrow p$
        \item[Neg] $\lnot(\lnot p)\Lleftarrow\Rrightarrow p$
        \item[Simp1] $p\land true\Lleftarrow\Rrightarrow p$
        \item[Assoc] $p \land (q \land r)\Lleftarrow\Rrightarrow (p \land q)
          \land r$
        \item[Dm] $\lnot(p\land q)\Lleftarrow\Rrightarrow \lnot \lor \lnot q$
        \item[Distr] $p \lor (q \land r)\Lleftarrow\Rrightarrow (p \lor q)
          \land (p \lor r)$
        \item[Contrapos] $p \Rightarrow q\Lleftarrow\Rrightarrow \lnot q
          \Rightarrow \lnot p$
        \item[Equiv] $p \Leftrightarrow q\Lleftarrow\Rrightarrow (p\Rightarrow
          q) \land (q \Rightarrow p)$
        \item[Simp2] $p \lor (p \land q)\Lleftarrow\Rrightarrow p$
      \end{itemize}

    \paragraph{Rules}
      \begin{enumerate}
        \item \textbf{Rule of substitution} substituting an equivalent for a
          subformula
        \item \textbf{Rule of transitivity} If $p\Lleftarrow\Rrightarrow q$ and
          $q\Lleftarrow\Rrightarrow r$, then $p\Lleftarrow\Rrightarrow r$
      \end{enumerate}

      \paragraph{Rule of Thumb}
      \begin{enumerate}
        \item Eliminate implication ($\Rightarrow$) and
          equivalence ($\Leftrightarrow$) using the law of implication, the law
          of equivalence and the contrapositive law backwards
        \item Simplify as soon as you can (simp 1, simp 2, idempotence,
          negation, kaw of contradiction, law of excluded middle)
        \item Sometimes use the various kinds of simplification backwards to
          prepare for using distributivity
      \end{enumerate}

    \subsection{Semantics}
      Transformational proof satisfies the following:
      \begin{enumerate}
        \item If $p\Lleftarrow\Rrightarrow q$ can be proved, then $p
          \leftrightarrow q$ (\textbf{soundness})
        \item If $p \leftrightarrow q$, then $p\Lleftarrow\Rrightarrow q$ can
          be proved (\textbf{completeness})
      \end{enumerate}
      Thus, transformational proof is sound and complete for propositional
      logic, and we use this to show the logical equivalence of two formulas.
      This way os oftenless tedious than the truth tables
