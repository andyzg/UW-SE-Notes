\chapter{Module 2 - Propositional Logic}
  \section{Definitions}
    \paragraph{Proof Theory} are methods that manipulate strings of symbols
    base on pattern matching. There may be multiple ways to prove a formula.

    \paragraph{Sound} If $p_1, p_2,\ldots,p_n \models q$ (proof), then $p_1,
    p_2,\ldots,p_n \vdash q$ (valid)

    \paragraph{Complete} If $p_1, p_2,\ldots,p_n \vdash q$ (valid) then
    $p_1,p_2,\ldots,p_n \models q$ (proof)

    \paragraph{Boolean Valuation} is a function $v$ from the set of
    formulas in propositional logic to the set $T_r$. Boolean valuation is
    also called a model or an interpretation.
    \begin{itemize}
      \item $v(false) = F, v(true) = T$
      \item $v(\lnot p) = NOT (v(p))$
      \item For the connectives:
      \begin{itemize}
        \item $v(p\land q) = v(p) AND v(q)$
        \item $v(p\lor q) = v(p) OR v(q)$
        \item $v(p\Rightarrow q) = v(p) IMP v(q)$
        \item $v(p\Leftrightarrow q) = v(p) IFF v(q)$
      \end{itemize}
    \end{itemize}

    \paragraph{Satisfiability} If a formula is \textbf{satisfiable}, then
    there exists a Boolean valuation $v$ such that $v(p) = T$

    \paragraph{Tautologies} A propositional formula $p$ is a
    \textbf{tautology} or \textbf{valid} if $v(p) = T$ for all Boolean
    valuations $v$. When a formula $q$ is a tautology, we write
    \begin{center}
       $\vdash q$
    \end{center}

    \paragraph{Logical Implication} A formula $p$ \textbf{logically
    implies} a formula $q$ iff for all Boolean valuations $v$, if for all
    premises $v(p_i) =
    T$, then $v(q) = T$, meaning
    \begin{center}
      $p \vdash q$ which is equivalent to $\vdash p \Rightarrow q$
    \end{center}

    \paragraph{Contradiction} A propositional formula $a$ is a
    \textbf{contradiction} if $v(a) = F$ for all Boolean valuations $v$

    \paragraph{Contingent} A \textbf{contingent} is one that is neither a
    \textbf{tautology} nor a \textbf{contradiction}

    \paragraph{Logical Equivalence} Two formulas are \textbf{logically
    equivalent} iff their equivalence is a tautology i.e. $v(p) = v(q)$ for
    all $v$.
    \begin{center}
      $p\leftrightarrow q$ which also means $\vdash p \Leftrightarrow q$
    \end{center}
    \begin{itemize}
      \item[$\leftrightarrow$] Logical equivalence
      \item[$\Leftrightarrow$] Material equivalence
    \end{itemize}

    \paragraph{Consistency} A collection of formulas is \textbf{consistent}
    if there exists a boolean valuation where all the formulas can be true
    simultaneously. If a set of formulas in the antecedent of an
    implication, they can be used to prove a contradiction.

    \paragraph{Transformational Proof Rules}
    \begin{itemize}
      \item[Comm] $p\land q \Lleftarrow\Rrightarrow q\land p$
      \item[Lem] $p\lor\lnot \Lleftarrow\Rrightarrow true$
      \item[Contr] $p\land\lnot p\Lleftarrow\Rrightarrow false$
      \item[Impl] $p \Rightarrow q\Lleftarrow\Rrightarrow \lnot p \lor q$
      \item[Idemp] $p \land p\Lleftarrow\Rrightarrow p$
      \item[Neg] $\lnot(\lnot p)\Lleftarrow\Rrightarrow p$
      \item[Simp1] $p\land true\Lleftarrow\Rrightarrow p$
      \item[Assoc] $p \land (q \land r)\Lleftarrow\Rrightarrow (p \land q)
        \land r$
      \item[Dm] $\lnot(p\land q)\Lleftarrow\Rrightarrow \lnot \lor \lnot q$
      \item[Distr] $p \lor (q \land r)\Lleftarrow\Rrightarrow (p \lor q)
        \land (p \lor r)$
      \item[Contrapos] $p \Rightarrow q\Lleftarrow\Rrightarrow \lnot q
        \Rightarrow \lnot p$
      \item[Equiv] $p \Leftrightarrow q\Lleftarrow\Rrightarrow (p\Rightarrow
        q) \land (q \Rightarrow p)$
      \item[Simp2] $p \lor (p \land q)\Lleftarrow\Rrightarrow p$
    \end{itemize}

    \paragraph{Rules}
      \begin{enumerate}
        \item \textbf{Rule of substitution} substituting an equivalent for a
          subformula
        \item \textbf{Rule of transitivity} If $p\Lleftarrow\Rrightarrow q$ and
          $q\Lleftarrow\Rrightarrow r$, then $p\Lleftarrow\Rrightarrow r$
      \end{enumerate}

    \paragraph{Literal} is a propositional letter or the negation of a
    proposition letter

    \paragraph{Conjunctive Normal Form} is a conjunction (AND) of clauses,
    where a clause is a dusjunction (OR) of literal or a single literal

    \paragraph{Disjunctive Normal Form} is a disjunction (OR) of clauses, where
    a clause is a conjunction (AND) of literals or a single literal

    \paragraph{Argument} is a collection of formulas, one of which reffered to
    as the conclusion, is justified by the others, referred to as the premises.

    \paragraph{Deductive} means the conclusion of an argument is wholly
    justified by the premises

    \paragraph{Inductive} arguments conclude more general new knowledge from a
    small number of particular facts or observations.

    \paragraph{Valid} means all Boolean valuations where the premises have the
    value T, the conclusion has the truth value T

    \paragraph{Invalid} means there is one Boolean valuation where premises is
    true but the conclusion is false

    \paragraph{Natural Deducation} is a collection of rules called inference
    rules which allows us to infer new formulas from given formulas.

    \paragraph{Inference rule} is a primitive valid argument form. Each
    inference rule enables the elimination or the introduction of a logical
    connective

    \paragraph{Syllogism} is a kind of logical argument that applies deductive
    reasoning to arrive at a conclusion based on two or more propositions that
    are asserted or assumed to be true.

    \paragraph{Summary of Natural Deduction Rules}
      \begin{itemize}
        \item $ p, q $ (and i) $ p \land q $
        \item $ p \and q $ (and e) $ p $
        \item $ p $ (or i) $ p \lor q $
        \item $ p \lor r $ (cases) $ case p \{ ... q \} case r \{ ... q \} q $
        \item  assume $ r $ \{ ... $ q $ \} (imp i) $ r \Rightarrow q $
        \item $ p \Rightarrow q, p $ (imp e) $ q $
        \item  disprove $r$ \{ ... false \} (raa) $ \lnot r $
        \item $ p, \lnot p $ (not e) $ q $
        \item $ p $ (not not i) $ \lnot \lnot p $
        \item $ \lnot \lnot p $ (not not e) $ p $
        \item $ p \Rightarrow q, q \Rightarrow p $ (iff i) $ p \Leftrightarrow
          q $
        \item $ p \Leftrightarrow q $ (iff e) $ p \Rightarrow q $
        \item $ p \lor q $, $ \lnot p $ (or e) $ q $
        \item (lem) $p \lor \lnot p$
      \end{itemize}

    \paragraph{Summary of Semantic Tableaux Rules}
      \begin{itemize}
        \item $ p \land q $ (and nb) $p$, $q$
        \item $ \lnot ( p \land q ) $ (not and br) 1. $ \lnot p $ 2. $ \lnot q $
        \item $ p \lor q $ (or br) 1. $p$ 2. $q$
        \item $ \lnot ( p \lor q ) $ (not or nb) $ \lnot p $, $ \lnot q $
        \item $ p \Rightarrow $ (imp br) 1. $ \lnot p $ 2. $ q $
        \item $ \lnot ( p \Rightarrow q ) $ (not imp nb) $p$, $\lnot q$
        \item $ \lnot \lnot p $ (not not nb) $ p $
        \item $ p \Leftrightarrow q $ (iff br) 1. $ p \land q $ 2. $ \lnot p
          \land \lnot q $
        \item $ \lnot ( p \Leftrightarrow q ) $ (not iff br) 1. $ p \land
          \lnot q $ 2. $ \lnot p \land q $
      \end{itemize}
