\chapter{Module 5 - Sets}
  \section{Definitions}
    \paragraph{Set} is a collection of element or members

    \paragraph{Set enumeration} List the item in the set

    \paragraph{Set comprehension} Define a set by using a predicate

    \paragraph{Z notation} $ \{ \langle term \rangle \bullet \langle signature
    \rangle | \langle predicate \rangle \} $

    \paragraph{Term} is a term in predicate logic i.e an expression using
    variables and functions that returns an object. Term can be omitted if it
    is just a variable and we have a signature

    \paragraph{Signature} lists the variables used in term and their types. If
    we are not using types, the signature can be omitted, but we need one of
    the term or the signature

    \paragraph{Predicate} is any wff formula in predicate logic with the
    variables used in term as free variables in the formula. If no predicate is
    needed, i.e the formula true would be used, then it can be omitted

    \paragraph{Empty set} is written $\emptyset$

    \paragraph{Universal Set} consists of all the objects of concern in any
    discussion. Often, the universal set is an appropriate choice for a type.
    It is denoted as $U$

    \paragraph{Single set} is a set consisting of only one element

    \paragraph{Power set} of a set is the set of all of its subsets. P is the
    function that returns the power set of a set. Usually we leave off the
    brackets around the argument of this function.

    \paragraph{Set Theory Summary}
    \begin{itemize}
      \item Types as sets: $ forall x : B \bullet P(x) \Leftrightarrow forall x
        \bullet x \in B \Rightarrow P(x) $
      \item Set comprehension: $ x \in \{ y \bullet y : S | P(y) \}
        \Leftrightarrow x \in S \land P(x) $
      \item Empty set: $ \forall x \bullet \lnot ( x \in \emptyset ) $
      \item Set equality: $ D == B \Leftrightarrow ( \forall x \bullet x \in D
        \Leftrightarrow x \in B ) $
      \item Subset: $ D \subseteq B \Leftrightarrow ( \forall x \bullet x \in D
        \Rightarrow x \in B ) $
      \item Proper Subset: $ D \subset B \Leftrightarrow D \subseteq B \land
        \lnot ( D == B ) $
      \item Power Set: $ PD == \{ B | B \subseteq D \} $
      \item Set Union: $ D \cup B == \{ x | x \in D \lor x \in B \} $
      \item Set intersection: $ D \cap B == \{ x | x \in D \land x \in B \} $
      \item Absolute Complement: $D' == \{ x | x \in U \land \lnot ( x \in D )
        \} $
      \item Set difference: $ D - B == \{ x | x \in D \land \lnot ( x \in B )
        \} $
    \end{itemize}

    \paragraph{Disjoint} Two sets D and B are disjoint if their intersection is
    empty
