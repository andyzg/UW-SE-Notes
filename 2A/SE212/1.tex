\chapter{Module 1 --- Sept 8}
  \section{Logic and Computation, module 1}
    \subsection{Contact}
      Amirhossein Vakili, avakili@uwaterloo.ca\\
      SE212 Mon/Wed 11:30 to 12:50 RCH 305\\
      Office Hours: Tues 2:30 to 3:30 DC2551

    \subsection{Course content}
      How do you know what a program is supposed to do?
      (specification/correctness)
      \begin{itemize}
        \item Inspection
        \item Testing
        \item \textbf{Formal verification}
      \end{itemize}

    \subsection{Logic}
      \paragraph{Formal verification} \hspace{0pt}~\\
      Logic is based on logical reasoning, also called `formal methods' or
      `computer-aided verification'. It checks the correctness of a program
      for all outputs. Since this takes a lot of effort, it is complementary to
      testing and inspection.

      \paragraph{Logic}  \hspace{0pt}~\\
      A logic consists of:
      \begin{itemize}
        \item \textit{syntax} \: What is an acceptable sentence
        \item \textit{semantics} \: What do the symbols and sentences in the
          language mean?
        \item \textit{proof theory} \: How do we construct valid proofs?
      \end{itemize}
      Logic provides a way to express knowledge precisely and to reason
      consequences of that knowledge

    \subsection{Course Outline}
      Four main topics:
      \begin{itemize}
        \item Propositional logic
        \item Predicate logic
        \item Set Theory and Specification
        \item Program correctness
      \end{itemize}

    \subsection{Marking Scheme}
      \begin{itemize}
        \item 20\% assignments (top 7 out of 8 assignments)
        \item 25\% Midterm exam
        \item 55\% Final exam
      \end{itemize}

