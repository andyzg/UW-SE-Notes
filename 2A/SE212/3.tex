\chapter{Module 3 - Predicate Logic}
  \section{Definitions}
    \paragraph{Predicate} is a symbol denoting the meaning of an attribute of
    an object or the meaning of a relationship between two or more objects

    \paragraph{Unary, binary, N ary predicate} takes 1, 2 or n number of
    objects as arguments

    \paragraph{Constant} is a symbol denoting a particular object

    \paragraph{Quantifiers}
      \begin{itemize}
        \item[forall] $ \forall $, predicate holds for all elements (every,
        all, for all). The formula must be true for all substitutions of an
        object in the domain for the quantified variable
        \item[exists] $ \exists $, predicate holds for some element (at least
        one, some, there exists). For an existentially quantified variable,
        the formula must be true for some substitution of an object in the
        domain for the quantified variable
      \end{itemize}

    \paragraph{Terms}
      \begin{itemize}
        \item Every constant is a term
        \item Every variable is a term
        \item If $t_1, t_2, t_3 ... t_n$ are terms and $f$ is an $n$ ary
        function symbol, then $f(t_1, t_2, ... t_n)$
        \item Nothing is a term
        \item Terms describe objects
      \end{itemize}

    \paragraph{Scope of a quantifier} is a subformula over which the quantifier
    applies in the given formula. Without brackets, we assume the scope of the
    quantifier extends to the right end of the formula

    \paragraph{Bound Variable} is an occurence of a variable that is within the
    scope of a quantifier over that variable. A variable is bound to the
    closest quantifier to the left of its name whose scope it is within. All
    variable occurrences that are bound to the same quantifier represent the
    same object

    \paragraph{Free}, a variable is free if it does not fall within the scope of
    a quantifier for that variable

    \paragraph{Closed} A wff is closed if it contains no free variables (we
    work with closed wff)

    \paragraph{Type} is a set of objects describing the possible values of a
    variable. These values are constant of that type. We assume types are non
    empty, i.e. at least one element of any type exists. We won't need types
    unless explicitly instructed.

    \paragraph{Satisfiability} A predicate logic is satisfiable iff there
    exists an interpretation I that satisfies the formula. An interpretation I
    satisfies a predicate logic formula A iff $M[[A]] = T$ when evaluted I

    \paragraph{Tautology} A predicate logic formula is a tautology iff every
    interpretation satisfies the formula, $\vdash$ P

    \paragraph{Variable capture} means a free variable becomes bound after the
    substitution

    \paragraph{Genuine varaible} is a free variable such that the universal
    quantification of it yields a formula that is true. It represents any value

    \paragraph{Unknown varaible} is a free variable such that the existential
    quantification of it yields a formula that is true. It represents a
    specific (but unknown) value

    \paragraph{Summary of Natural Deduction for Predicate Logic}
    \begin{itemize}
      \item for every $x_g$ \{ ... $P[x_g / x]$ \} (forall i) $\forall x
        \bullet P$
      \item $\forall x \bullet P$ (forall e) $P[t/x]$
      \item $P[t/x]$ (exists i) $\exists x \bullet P$
      \item $\exists x \bullet P$, for some $x_u$ $P[x_u/x]$ \{ ... Q \}
        (exists e) Q, $x_u$ does not appear in Q
    \end{itemize}

    \paragraph{Summary of Semantic Tableau for Predicate Logic}
    \begin{itemize}
      \item $\forall x \bullet P(x)$ (forall nb) $P(t)$, where $t$ is a term
        that is a legal substitution
      \item $\lnot(\forall x \bullet P(x))$ (not forall nb) $\exists x \bullet
        \lnot P(x)$
      \item $\exists x \bullet P(x)$ (exists nb) $P(y)$, where $y$ is a
        variable that has not been used in the tableau so far
      \item $\lnot(\exists x \bullet P(x))$ (not exists nb) $\forall x \bullet
        \lnot P(x)$
    \end{itemize}
