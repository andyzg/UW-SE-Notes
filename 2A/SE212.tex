\documentclass[12pt]{report}
\usepackage{dcolumn}
\usepackage{listings}
\newcolumntype{d}[1]{D{.}{\cdot}{#1} }
\usepackage[utf8]{inputenc}
\usepackage[margin=2cm]{geometry}
\usepackage{amsfonts}
\usepackage{amsmath}
\usepackage{amssymb}
\usepackage{enumitem}
\setlist{nolistsep}
\usepackage[hidelinks]{hyperref}

\title{SE212}
\author{{Andy Zhang}}
\date{{Fall 2014}}
\begin{document}
\maketitle
\tableofcontents
\chapter{Module 2 - Propositional Logic}
  \section{Definitions}
    \paragraph{Proof Theory} are methods that manipulate strings of symbols
    base on pattern matching. There may be multiple ways to prove a formula.

    \paragraph{Sound} If $p_1, p_2,\ldots,p_n \models q$ (proof), then $p_1,
    p_2,\ldots,p_n \vdash q$ (valid)

    \paragraph{Complete} If $p_1, p_2,\ldots,p_n \vdash q$ (valid) then
    $p_1,p_2,\ldots,p_n \models q$ (proof)

    \paragraph{Boolean Valuation} is a function $v$ from the set of
    formulas in propositional logic to the set $T_r$. Boolean valuation is
    also called a model or an interpretation.
    \begin{itemize}
      \item $v(false) = F, v(true) = T$
      \item $v(\lnot p) = NOT (v(p))$
      \item For the connectives:
      \begin{itemize}
        \item $v(p\land q) = v(p) AND v(q)$
        \item $v(p\lor q) = v(p) OR v(q)$
        \item $v(p\Rightarrow q) = v(p) IMP v(q)$
        \item $v(p\Leftrightarrow q) = v(p) IFF v(q)$
      \end{itemize}
    \end{itemize}

    \paragraph{Satisfiability} If a formula is \textbf{satisfiable}, then
    there exists a Boolean valuation $v$ such that $v(p) = T$

    \paragraph{Tautologies} A propositional formula $p$ is a
    \textbf{tautology} or \textbf{valid} if $v(p) = T$ for all Boolean
    valuations $v$. When a formula $q$ is a tautology, we write
    \begin{center}
       $\vdash q$
    \end{center}

    \paragraph{Logical Implication} A formula $p$ \textbf{logically
    implies} a formula $q$ iff for all Boolean valuations $v$, if for all
    premises $v(p_i) =
    T$, then $v(q) = T$, meaning
    \begin{center}
      $p \vdash q$ which is equivalent to $\vdash p \Rightarrow q$
    \end{center}

    \paragraph{Contradiction} A propositional formula $a$ is a
    \textbf{contradiction} if $v(a) = F$ for all Boolean valuations $v$

    \paragraph{Contingent} A \textbf{contingent} is one that is neither a
    \textbf{tautology} nor a \textbf{contradiction}

    \paragraph{Logical Equivalence} Two formulas are \textbf{logically
    equivalent} iff their equivalence is a tautology i.e. $v(p) = v(q)$ for
    all $v$.
    \begin{center}
      $p\leftrightarrow q$ which also means $\vdash p \Leftrightarrow q$
    \end{center}
    \begin{itemize}
      \item[$\leftrightarrow$] Logical equivalence
      \item[$\Leftrightarrow$] Material equivalence
    \end{itemize}

    \paragraph{Consistency} A collection of formulas is \textbf{consistent}
    if there exists a boolean valuation where all the formulas can be true
    simultaneously. If a set of formulas in the antecedent of an
    implication, they can be used to prove a contradiction.

    \paragraph{Transformational Proof Rules}
    \begin{itemize}
      \item[Comm] $p\land q \Lleftarrow\Rrightarrow q\land p$
      \item[Lem] $p\lor\lnot \Lleftarrow\Rrightarrow true$
      \item[Contr] $p\land\lnot p\Lleftarrow\Rrightarrow false$
      \item[Impl] $p \Rightarrow q\Lleftarrow\Rrightarrow \lnot p \lor q$
      \item[Idemp] $p \land p\Lleftarrow\Rrightarrow p$
      \item[Neg] $\lnot(\lnot p)\Lleftarrow\Rrightarrow p$
      \item[Simp1] $p\land true\Lleftarrow\Rrightarrow p$
      \item[Assoc] $p \land (q \land r)\Lleftarrow\Rrightarrow (p \land q)
        \land r$
      \item[Dm] $\lnot(p\land q)\Lleftarrow\Rrightarrow \lnot \lor \lnot q$
      \item[Distr] $p \lor (q \land r)\Lleftarrow\Rrightarrow (p \lor q)
        \land (p \lor r)$
      \item[Contrapos] $p \Rightarrow q\Lleftarrow\Rrightarrow \lnot q
        \Rightarrow \lnot p$
      \item[Equiv] $p \Leftrightarrow q\Lleftarrow\Rrightarrow (p\Rightarrow
        q) \land (q \Rightarrow p)$
      \item[Simp2] $p \lor (p \land q)\Lleftarrow\Rrightarrow p$
    \end{itemize}

    \paragraph{Rules}
      \begin{enumerate}
        \item \textbf{Rule of substitution} substituting an equivalent for a
          subformula
        \item \textbf{Rule of transitivity} If $p\Lleftarrow\Rrightarrow q$ and
          $q\Lleftarrow\Rrightarrow r$, then $p\Lleftarrow\Rrightarrow r$
      \end{enumerate}

    \paragraph{Literal} is a propositional letter or the negation of a
    proposition letter

    \paragraph{Conjunctive Normal Form} is a conjunction (AND) of clauses,
    where a clause is a dusjunction (OR) of literal or a single literal

    \paragraph{Disjunctive Normal Form} is a disjunction (OR) of clauses, where
    a clause is a conjunction (AND) of literals or a single literal

    \paragraph{Argument} is a collection of formulas, one of which reffered to
    as the conclusion, is justified by the others, referred to as the premises.

    \paragraph{Deductive} means the conclusion of an argument is wholly
    justified by the premises

    \paragraph{Inductive} arguments conclude more general new knowledge from a
    small number of particular facts or observations.

    \paragraph{Valid} means all Boolean valuations where the premises have the
    value T, the conclusion has the truth value T

    \paragraph{Invalid} means there is one Boolean valuation where premises is
    true but the conclusion is false

    \paragraph{Natural Deducation} is a collection of rules called inference
    rules which allows us to infer new formulas from given formulas.

    \paragraph{Inference rule} is a primitive valid argument form. Each
    inference rule enables the elimination or the introduction of a logical
    connective

    \paragraph{Syllogism} is a kind of logical argument that applies deductive
    reasoning to arrive at a conclusion based on two or more propositions that
    are asserted or assumed to be true.

    \paragraph{Summary of Natural Deduction Rules}
      \begin{itemize}
        \item $ p, q $ (and i) $ p \land q $
        \item $ p \and q $ (and e) $ p $
        \item $ p $ (or i) $ p \lor q $
        \item $ p \lor r $ (cases) $ case p \{ ... q \} case r \{ ... q \} q $
        \item  assume $ r $ \{ ... $ q $ \} (imp i) $ r \Rightarrow q $
        \item $ p \Rightarrow q, p $ (imp e) $ q $
        \item  disprove $r$ \{ ... false \} (raa) $ \lnot r $
        \item $ p, \lnot p $ (not e) $ q $
        \item $ p $ (not not i) $ \lnot \lnot p $
        \item $ \lnot \lnot p $ (not not e) $ p $
        \item $ p \Rightarrow q, q \Rightarrow p $ (iff i) $ p \Leftrightarrow
          q $
        \item $ p \Leftrightarrow q $ (iff e) $ p \Rightarrow q $
        \item $ p \lor q $, $ \lnot p $ (or e) $ q $
        \item (lem) $p \lor \lnot p$
      \end{itemize}

    \paragraph{Summary of Semantic Tableaux Rules}
      \begin{itemize}
        \item $ p \land q $ (and nb) $p$, $q$
        \item $ \lnot ( p \land q ) $ (not and br) 1. $ \lnot p $ 2. $ \lnot q $
        \item $ p \lor q $ (or br) 1. $p$ 2. $q$
        \item $ \lnot ( p \lor q ) $ (not or nb) $ \lnot p $, $ \lnot q $
        \item $ p \Rightarrow $ (imp br) 1. $ \lnot p $ 2. $ q $
        \item $ \lnot ( p \Rightarrow q ) $ (not imp nb) $p$, $\lnot q$
        \item $ \lnot \lnot p $ (not not nb) $ p $
        \item $ p \Leftrightarrow q $ (iff br) 1. $ p \land q $ 2. $ \lnot p
          \land \lnot q $
        \item $ \lnot ( p \Leftrightarrow q ) $ (not iff br) 1. $ p \land
          \lnot q $ 2. $ \lnot p \land q $
      \end{itemize}

\chapter{Module 3 - Predicate Logic}
  \section{Definitions}
    \paragraph{Predicate} is a symbol denoting the meaning of an attribute of
    an object or the meaning of a relationship between two or more objects

    \paragraph{Unary, binary, N ary predicate} takes 1, 2 or n number of
    objects as arguments

    \paragraph{Constant} is a symbol denoting a particular object

    \paragraph{Quantifiers}
      \begin{itemize}
        \item[forall] $ \forall $, predicate holds for all elements (every,
        all, for all). The formula must be true for all substitutions of an
        object in the domain for the quantified variable
        \item[exists] $ \exists $, predicate holds for some element (at least
        one, some, there exists). For an existentially quantified variable,
        the formula must be true for some substitution of an object in the
        domain for the quantified variable
      \end{itemize}

    \paragraph{Terms}
      \begin{itemize}
        \item Every constant is a term
        \item Every variable is a term
        \item If $t_1, t_2, t_3 ... t_n$ are terms and $f$ is an $n$ ary
        function symbol, then $f(t_1, t_2, ... t_n)$
        \item Nothing is a term
        \item Terms describe objects
      \end{itemize}

    \paragraph{Scope of a quantifier} is a subformula over which the quantifier
    applies in the given formula. Without brackets, we assume the scope of the
    quantifier extends to the right end of the formula

    \paragraph{Bound Variable} is an occurence of a variable that is within the
    scope of a quantifier over that variable. A variable is bound to the
    closest quantifier to the left of its name whose scope it is within. All
    variable occurrences that are bound to the same quantifier represent the
    same object

    \paragraph{Free}, a variable is free if it does not fall within the scope of
    a quantifier for that variable

    \paragraph{Closed} A wff is closed if it contains no free variables (we
    work with closed wff)

    \paragraph{Type} is a set of objects describing the possible values of a
    variable. These values are constant of that type. We assume types are non
    empty, i.e. at least one element of any type exists. We won't need types
    unless explicitly instructed.

    \paragraph{Satisfiability} A predicate logic is satisfiable iff there
    exists an interpretation I that satisfies the formula. An interpretation I
    satisfies a predicate logic formula A iff $M[[A]] = T$ when evaluted I

    \paragraph{Tautology} A predicate logic formula is a tautology iff every
    interpretation satisfies the formula, $\vdash$ P

    \paragraph{Variable capture} means a free variable becomes bound after the
    substitution

    \paragraph{Genuine varaible} is a free variable such that the universal
    quantification of it yields a formula that is true. It represents any value

    \paragraph{Unknown varaible} is a free variable such that the existential
    quantification of it yields a formula that is true. It represents a
    specific (but unknown) value

    \paragraph{Summary of Natural Deduction for Predicate Logic}
    \begin{itemize}
      \item for every $x_g$ \{ ... $P[x_g / x]$ \} (forall i) $\forall x
        \bullet P$
      \item $\forall x \bullet P$ (forall e) $P[t/x]$
      \item $P[t/x]$ (exists i) $\exists x \bullet P$
      \item $\exists x \bullet P$, for some $x_u$ $P[x_u/x]$ \{ ... Q \}
        (exists e) Q, $x_u$ does not appear in Q
    \end{itemize}

    \paragraph{Summary of Semantic Tableau for Predicate Logic}
    \begin{itemize}
      \item $\forall x \bullet P(x)$ (forall nb) $P(t)$, where $t$ is a term
        that is a legal substitution
      \item $\lnot(\forall x \bullet P(x))$ (not forall nb) $\exists x \bullet
        \lnot P(x)$
      \item $\exists x \bullet P(x)$ (exists nb) $P(y)$, where $y$ is a
        variable that has not been used in the tableau so far
      \item $\lnot(\exists x \bullet P(x))$ (not exists nb) $\forall x \bullet
        \lnot P(x)$
    \end{itemize}

\chapter{Module 4 - Theories}
  \section{Definitions}
    \paragraph{Enthymeme} is an argument that contains a hidden premise, that
    is, an argument that contains unstated premises that are obviously true

    \paragraph{Theory}
    \begin{itemize}
      \item A set of statements or principles devised to explain a group of
      facts or phenomena, especially one that has been repeatedly tested or is
      widely accepted and can be used to make predictions about natural
      phenomena
      \item A set of theorems that constitute a systematic view of a branch of
      mathematics
    \end{itemize}

    \paragraph{Leibnizs Law} if $t_1 == t_2$ is a theorem, then so is $P[ t_1 / x ]
    \Leftrightarrow P[ t_2 / x ]$\\
    Lebinizs Law is generally referred to as the ability to substitute equals
    for equals

    \paragraph{Normal interpretations} are interpretations in which the symbol
    == is interpreted as equality on the objects of the domain.

    \paragraph{Induction}
      \begin{itemize}
        \item $P(0)$ is called the base case
        \item $P(k)$ is called the induction hypothesis
        \item $\forall k \wedge P(k) \Rightarrow P(suc(k))$ is called the
        induction proof or inductive step
      \end{itemize}

    \paragraph{Deduction} is showing a conclusion follows frmo the stated
    premises using rules of inference

    \paragraph{Philosophical induction} is the process of deriving general
    principles from particular observations

    \paragraph{Recursive function} is one that is defined in terms of itself
    and certain terminating clauses

    \paragraph{Satisfiable Modulo Theories}
      is a decision problem for logical formulas with respect to combinations
      of background theories experessed in classical first order logic with
      equality.

    \paragraph{Rule of Thumb} when formalizing a set of sentences.
    \begin{enumerate}
      \item Identify all possible logical connectives. Recall the heuristic
        from propositional logic: pick the smallest phrases without and or if
        then etc.
      \item Identify what might be quantified, implicit quantifiers, constants
      \item Identify the possible predicates and functinos. The quantified
        variables must be arguments to the predicates
      \item Look for similar phrases to determine how much detail you need to
        provide
      \item Think of an outline of how your proof will proceed.
    \end{enumerate}

    \paragraph{Rule of Thumb} for types
    \begin{enumerate}
      \item If a unary predicate is used in all premises and conclusions as
        part of the antecedent of a universally quantified formula, it is a
        good candidate to make a type
      \item If a unary predicate is used in all premises as one conjuct of an
        existentially quantified formula, it is a good candidate to make a type
    \end{enumerate}

\chapter{Module 5 - Sets}
  \section{Definitions}
    \paragraph{Set} is a collection of element or members

    \paragraph{Set enumeration} List the item in the set

    \paragraph{Set comprehension} Define a set by using a predicate

    \paragraph{Z notation} $ \{ \langle term \rangle \bullet \langle signature
    \rangle | \langle predicate \rangle \} $

    \paragraph{Term} is a term in predicate logic i.e an expression using
    variables and functions that returns an object. Term can be omitted if it
    is just a variable and we have a signature

    \paragraph{Signature} lists the variables used in term and their types. If
    we are not using types, the signature can be omitted, but we need one of
    the term or the signature

    \paragraph{Predicate} is any wff formula in predicate logic with the
    variables used in term as free variables in the formula. If no predicate is
    needed, i.e the formula true would be used, then it can be omitted

    \paragraph{Empty set} is written $\emptyset$

    \paragraph{Universal Set} consists of all the objects of concern in any
    discussion. Often, the universal set is an appropriate choice for a type.
    It is denoted as $U$

    \paragraph{Single set} is a set consisting of only one element

    \paragraph{Power set} of a set is the set of all of its subsets. P is the
    function that returns the power set of a set. Usually we leave off the
    brackets around the argument of this function.

    \paragraph{Set Theory Summary}
    \begin{itemize}
      \item Types as sets: $ forall x : B \bullet P(x) \Leftrightarrow forall x
        \bullet x \in B \Rightarrow P(x) $
      \item Set comprehension: $ x \in \{ y \bullet y : S | P(y) \}
        \Leftrightarrow x \in S \land P(x) $
      \item Empty set: $ \forall x \bullet \lnot ( x \in \emptyset ) $
      \item Set equality: $ D == B \Leftrightarrow ( \forall x \bullet x \in D
        \Leftrightarrow x \in B ) $
      \item Subset: $ D \subseteq B \Leftrightarrow ( \forall x \bullet x \in D
        \Rightarrow x \in B ) $
      \item Proper Subset: $ D \subset B \Leftrightarrow D \subseteq B \land
        \lnot ( D == B ) $
      \item Power Set: $ PD == \{ B | B \subseteq D \} $
      \item Set Union: $ D \cup B == \{ x | x \in D \lor x \in B \} $
      \item Set intersection: $ D \cap B == \{ x | x \in D \land x \in B \} $
      \item Absolute Complement: $D' == \{ x | x \in U \land \lnot ( x \in D )
        \} $
      \item Set difference: $ D - B == \{ x | x \in D \land \lnot ( x \in B )
        \} $
    \end{itemize}

    \paragraph{Disjoint} Two sets D and B are disjoint if their intersection is
    empty

\end{document}
