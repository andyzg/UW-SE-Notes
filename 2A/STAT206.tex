\documentclass[12pt]{report}
\usepackage{dcolumn}
\usepackage{listings}
\newcolumntype{d}[1]{D{.}{\cdot}{#1} }
\usepackage[utf8]{inputenc}
\usepackage[margin=2cm]{geometry}
\usepackage{amsfonts}
\usepackage{amsmath}
\usepackage{amssymb}
\usepackage{enumitem}
\setlist{nolistsep}
\usepackage[hidelinks]{hyperref}

\title{STAT206}
\author{{Andy Zhang}}
\date{{Fall 2014}}
\begin{document}
\maketitle
\tableofcontents
\chapter{Introduction}
  \section{Statistics}
    \subsection{Definitions}
      \paragraph{Statistics}
      Collection, organization, analysis,
      interpretation and presentation of data. It is also defined as the
      quantification of uncertainty.

      \paragraph{Unit} A single element, usually a person or object, whose
      characteristics are of interest. Ex: A student enrolled in the course.

      \paragraph{Population} The set of all units which are of interest. Ex:
      All students enrolled in the course

      \paragraph{Variable} A measurement of the characteristic of interest
      from a unit. Ex: Number of Canadian provinces visited by a student

      \paragraph{Sample} A subset of units from the population for which
      measurements of the desired variable are actually made. Ex: 29 students
      chosen from the class

      \paragraph{Descriptive Statistics} Summarize the data in the sample, both
      graphically and numerically

      \paragraph{Inferential statistics} USe the sample data to estimate an
      attribute of the population. Include a quantification of uncertainty

      \paragraph{Sampling Error} An error which occurs due to the uncertainty
      in randomly selecting a sample.

      \paragraph{Study error} A systematic error which occurs because the
      sample does not accurately represent the population

    \subsection{Process}
      Identify the problem of interest
      \begin{itemize}
        \item Who or what do you want to learn about?
        \begin{itemize}
          \item Define the \textbf{population} of interest
          \item Individual elements of the population are called \textbf{units}
        \end{itemize}
        \item What research question would you like answered?
        \begin{itemize}
          \item Define your \textbf{hypothesis}
        \end{itemize}
      \end{itemize}
      Plan the data collection
      \begin{itemize}
        \item How will you select a subset of \textbf{units} from the
        \textbf{population} to be in your \textbf{sample}?
        \begin{itemize}
          \item How large will the \textbf{sample} be?
        \end{itemize}
        \item What is (are) the \textbf{variable (s)} of interest?
        \begin{itemize}
          \item How will you measure it (them)?
        \end{itemize}
      \end{itemize}
      Analyze the data
      \begin{itemize}
        \item Graph the data --- histogram, scatter-plot, etc
        \item Compute \textbf{Descritive statistics} --- e.g.\ sample mean,
          sample variance, etc.
        \item Compute \textbf{Inferential statistics} --- e.g.\ confidence
        intervals, hypothesis tests about population \textbf{parameters}
          \begin{itemize}
            \item Inferential statistics include a quantification of the
            sampling error
          \end{itemize}
      \end{itemize}
      Draw conclusions
      \begin{itemize}
        \item Use the results of your analysis to address the original research
        question
        \item Address limitations of the study, especially any potential
        systematic \textbf{study errors}
      \end{itemize}

    \subsection{Data Types}
      \paragraph{Categorical Variable} A qualitative measure. Each unit belongs
      to \textbf{one of K} possible classes.

      \paragraph{Discrete variable} A quantitative measure. Each unit's
      measurement can take on one of a \textbf{countable} number of possible
      values

      \paragraph{Continuous variable} A quantitative measure. Each unit's
      measurement can take on an \textbf{uncountable} number of possible values,
      usually some interval of real numbers

    \subsection{(Grouped) Frequency Tables}
      \begin{itemize}
        \item Display the number of units which are in each class
        \item Discrete / Continuous variables are grouped into classes
        \item In the case of numerical variables, there is a loss of
          information
      \end{itemize}
      See more: \url{http://en.wikipedia.org/wiki/Stem-and-leaf_display}

    \subsection{Stem and Leaf Plot}
      \begin{itemize}
        \item A \textbf{stem-and-leaf plot} is a way to summarize a relatively
          \textbf{small} data set, without the loss of information that occurs
          with a frequency table
        \item Left is possible \textbf{first} digits, right is remaining digits
          in ascending order
      \end{itemize}
      See more: \url{http://en.wikipedia.org/wiki/Stem-and-leaf_display}
    \subsection{Bar Chart}
      \begin{itemize}
        \item Bar charts are used to graphically display information from
          categorical variables
      \end{itemize}
      See more: \url{http://en.wikipedia.org/wiki/Bar_chart}

    \subsection{Histogram}
      \begin{itemize}
        \item A histogram is similar to a bar chart, but it's for numerical
          data
        \item The range is divided in distinct classes, and each observation is
          assigned to exactly one class
        \item Histogram shows frequency of observations in each class
      \end{itemize}
     See more: \url{http://en.wikipedia.org/wiki/Histogram}

      \begin{itemize}
        \item If class ranges are not same length, we can use density histogram
          instead
        \item When interpreting a density histogram, it is the area that is
          meaningful
        \item Height is $ height = \frac{relative frequency}{width} =
          \frac{frequency}{width * n} $
      \end{itemize}
      See more \url{http://en.wikipedia.org/wiki/Histogram}
    \subsection{Measures of Centrality}
      \begin{itemize}
        \item The \textbf{sample mean} of a set of $n$ values, $x_1, x_2,
          x_3,\ldots, x_n$ denoted by $\bar{x}$ is $\bar{x} =
          \frac{\sum_{i=1}^{n} x_i}{n}$
        \item The \textbf{median} is the number $x^*$ such that half of the
          observed values are below $x^*$ and half are above
        \item If after writing our values in ascending order, we donte the
          $i^{th}$ value as $x_{(i)}$, then\\
          \[
              x^* = \left\{
              \begin{array}{l l}
                x_{(\frac{n+1}{2})} & \quad \text{if $n$ is odd}\\
                x_{(\frac{n}{2})}+x_{(\frac{n+2}{2})}& \quad \text{if $n$ is
              even}
              \end{array} \right.
          \]
      \end{itemize}

    \subsection{Measures of Variability}
      Measures of variability
      \begin{itemize}
        \item The \textbf{sample variance} of a set of values $x_1, x_2,
          x_3,\ldots, x_n$ denoted by $s^2$ is
          \begin{center}
            $s^2 = \frac{\sum_{i=1}^{n}{(x_i-\bar{x})}^2}{n-1}$
          \end{center}
        \item The \textbf{sample standard deviation} denoted $s$, is the square
          root of the sample variance
        \item The \textbf{range} of the set is the difference between the
          maximum and minimum value
          \begin{center}
            $range = x_{(n)} - x_{(1)}$
          \end{center}
      \end{itemize}

    \subsection{Box Plot}
      \begin{itemize}
        \item The box indicates the middle 50\% of the observations, i.e.\ the
          second and third quartiles
        \item The line through the box indicates the median observation
        \item The whiskers indicate the highest and lowest observations
      \end{itemize}
      See more: \url{http://en.wikipedia.org/wiki/Box_plot}

\end{document}
