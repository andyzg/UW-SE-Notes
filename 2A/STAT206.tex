\documentclass[12pt]{report}
\usepackage{dcolumn}
\usepackage{listings}
\newcolumntype{d}[1]{D{.}{\cdot}{#1} }
\usepackage[utf8]{inputenc}
\usepackage[margin=2cm]{geometry}
\usepackage{amsfonts}
\usepackage{amsmath}
\usepackage{amssymb}
\usepackage{enumitem}
\setlist{nolistsep}
\usepackage[hidelinks]{hyperref}

\title{STAT206}
\author{{Andy Zhang}}
\date{{Fall 2014}}
\begin{document}
\maketitle
\tableofcontents
\chapter{Introduction}
  \section{Statistics}
    \subsection{Definitions}
      \paragraph{Statistics}
      Collection, organization, analysis,
      interpretation and presentation of data. It is also defined as the
      quantification of uncertainty.

      \paragraph{Unit} A single element, usually a person or object, whose
      characteristics are of interest. Ex: A student enrolled in the course.

      \paragraph{Population} The set of all units which are of interest. Ex:
      All students enrolled in the course

      \paragraph{Variable} A measurement of the characteristic of interest
      from a unit. Ex: Number of Canadian provinces visited by a student

      \paragraph{Sample} A subset of units from the population for which
      measurements of the desired variable are actually made. Ex: 29 students
      chosen from the class

      \paragraph{Descriptive Statistics} Summarize the data in the sample, both
      graphically and numerically

      \paragraph{Inferential statistics} USe the sample data to estimate an
      attribute of the population. Include a quantification of uncertainty

      \paragraph{Sampling Error} An error which occurs due to the uncertainty
      in randomly selecting a sample.

      \paragraph{Study error} A systematic error which occurs because the
      sample does not accurately represent the population

    \subsection{Process}
      Identify the problem of interest
      \begin{itemize}
        \item Who or what do you want to learn about?
        \begin{itemize}
          \item Define the \textbf{population} of interest
          \item Individual elements of the population are called \textbf{units}
        \end{itemize}
        \item What research question would you like answered?
        \begin{itemize}
          \item Define your \textbf{hypothesis}
        \end{itemize}
      \end{itemize}
      Plan the data collection
      \begin{itemize}
        \item How will you select a subset of \textbf{units} from the
        \textbf{population} to be in your \textbf{sample}?
        \begin{itemize}
          \item How large will the \textbf{sample} be?
        \end{itemize}
        \item What is (are) the \textbf{variable (s)} of interest?
        \begin{itemize}
          \item How will you measure it (them)?
        \end{itemize}
      \end{itemize}
      Analyze the data
      \begin{itemize}
        \item Graph the data --- histogram, scatter-plot, etc
        \item Compute \textbf{Descritive statistics} --- e.g.\ sample mean,
          sample variance, etc.
        \item Compute \textbf{Inferential statistics} --- e.g.\ confidence
        intervals, hypothesis tests about population \textbf{parameters}
          \begin{itemize}
            \item Inferential statistics include a quantification of the
            sampling error
          \end{itemize}
      \end{itemize}
      Draw conclusions
      \begin{itemize}
        \item Use the results of your analysis to address the original research
        question
        \item Address limitations of the study, especially any potential
        systematic \textbf{study errors}
      \end{itemize}

    \subsection{Data Types}
      \paragraph{Categorical Variable} A qualitative measure. Each unit belongs
      to \textbf{one of K} possible classes.

      \paragraph{Discrete variable} A quantitative measure. Each unit's
      measurement can take on one of a \textbf{countable} number of possible
      values

      \paragraph{Continuous variable} A quantitative measure. Each unit's
      measurement can take on an \textbf{uncountable} number of possible values,
      usually some interval of real numbers

    \subsection{(Grouped) Frequency Tables}
      \begin{itemize}
        \item Display the number of units which are in each class
        \item Discrete / Continuous variables are grouped into classes
        \item In the case of numerical variables, there is a loss of
          information
      \end{itemize}
      See more: \url{http://en.wikipedia.org/wiki/Stem-and-leaf_display}

    \subsection{Stem and Leaf Plot}
      \begin{itemize}
        \item A \textbf{stem-and-leaf plot} is a way to summarize a relatively
          \textbf{small} data set, without the loss of information that occurs
          with a frequency table
        \item Left is possible \textbf{first} digits, right is remaining digits
          in ascending order
      \end{itemize}
      See more: \url{http://en.wikipedia.org/wiki/Stem-and-leaf_display}
    \subsection{Bar Chart}
      \begin{itemize}
        \item Bar charts are used to graphically display information from
          categorical variables
      \end{itemize}
      See more: \url{http://en.wikipedia.org/wiki/Bar_chart}

    \subsection{Histogram}
      \begin{itemize}
        \item A histogram is similar to a bar chart, but it's for numerical
          data
        \item The range is divided in distinct classes, and each observation is
          assigned to exactly one class
        \item Histogram shows frequency of observations in each class
      \end{itemize}
     See more: \url{http://en.wikipedia.org/wiki/Histogram}

      \begin{itemize}
        \item If class ranges are not same length, we can use density histogram
          instead
        \item When interpreting a density histogram, it is the area that is
          meaningful
        \item Height is $ height = \frac{relative frequency}{width} =
          \frac{frequency}{width * n} $
      \end{itemize}
      See more \url{http://en.wikipedia.org/wiki/Histogram}
    \subsection{Measures of Centrality}
      \begin{itemize}
        \item The \textbf{sample mean} of a set of $n$ values, $x_1, x_2,
          x_3,\ldots, x_n$ denoted by $\bar{x}$ is $\bar{x} =
          \frac{\sum_{i=1}^{n} x_i}{n}$
        \item The \textbf{median} is the number $x^*$ such that half of the
          observed values are below $x^*$ and half are above
        \item If after writing our values in ascending order, we donte the
          $i^{th}$ value as $x_{(i)}$, then\\
          \[
              x^* = \left\{
              \begin{array}{l l}
                x_{(\frac{n+1}{2})} & \quad \text{if $n$ is odd}\\
                x_{(\frac{n}{2})}+x_{(\frac{n+2}{2})}& \quad \text{if $n$ is
              even}
              \end{array} \right.
          \]
      \end{itemize}

    \subsection{Measures of Variability}
      Measures of variability
      \begin{itemize}
        \item The \textbf{sample variance} of a set of values $x_1, x_2,
          x_3,\ldots, x_n$ denoted by $s^2$ is
          \begin{center}
            $s^2 = \frac{\sum_{i=1}^{n}{(x_i-\bar{x})}^2}{n-1}$
          \end{center}
        \item The \textbf{sample standard deviation} denoted $s$, is the square
          root of the sample variance
        \item The \textbf{range} of the set is the difference between the
          maximum and minimum value
          \begin{center}
            $range = x_{(n)} - x_{(1)}$
          \end{center}
      \end{itemize}

    \subsection{Box Plot}
      \begin{itemize}
        \item The box indicates the middle 50\% of the observations, i.e.\ the
          second and third quartiles
        \item The line through the box indicates the median observation
        \item The whiskers indicate the highest and lowest observations
      \end{itemize}
      See more: \url{http://en.wikipedia.org/wiki/Box_plot}

\chapter{Probability}
  \section{Definitions}
    \paragraph{Probability} measure the uncertainty associated with an event.
    An event is something that might occur
    \begin{itemize}
      \item Classical: $\frac{Number of ways event can occur}{Total number of
      equally likely outcomes}$
      \item Relative Frequency: Proportion of times the event occurs, as the
      number of trials approaches infinity
      \item Subjective: Estimates of probability that the event occurs, based
      on subjective opinion
    \end{itemize}

    \paragraph{Experiment} is a repeatable phenomenon or process
    \paragraph{Trial} is a single repetition of an experiment
    \paragraph{Sample Space}, S, is the set of distinct outcomes for an
    experiment or process
    \paragraph{Discrete} A sample space is discrete if it has a finite or
    countably infinite number of simple events. Otherwise it is non discrete or
    continuous

    \paragraph{Mutually Exclusive} means two events never occur simultaneously

    \paragraph{Complement} of an event and an event are always mutually
    exclusive

    \paragraph{Uniform distribution} The total probability is uniformly
    distributed among all possible outcomes

    \paragraph{Permutation} is the number of ways to arrange $r$ out of $n$
    objects: $ n ^ {(r)} = \frac{n!}{(n-r)!} = n(n-1)(n-2) ... (n-r+1)$

    \paragraph{Combinations} If we don't care about the order of objects, but
    just which objects are chosen, the number of ways to choose $r$ out of $n$
    items is $\binom{n}{r} = \frac{n!}{r!(n-r)!}$

    \paragraph{Set Operations}
    \begin{itemize}
      \item $AB$ or $A\cap B$ is the intersection of two
      events.
      \item$A \cup B$ is the union of two events.
      \item $\bar{A}$ is the complement of A, not event A
    \end{itemize}

    \paragraph{Conditional} The probability of event A, conditional on the
    occurence of event B, denoted by $P(A|B)$ is $P(A|B) = \frac{P(A \cap
    B)}{P(B)}$, $P(B) \neq 0$

    \paragraph{Independent} Two events are said to be independent iff $P(A \cap
    B) = P(A)P(B)$. This implies that $P(A|B) = P(A)$, $P(B|A) = P(B)$. In
    other words, events A and B are independent if whether B occurs does not
    influence whether A occurs, and vice versa

    \paragraph{Bayes' Theorem} Suppose A and B are any two events in S, then
    $P(B|A) = \frac{P(A|B)P(B)}{P(A|B)P(B) + P(A|\bar{B})P(\bar{B})}$

\chapter{Random Variables}
  \section{Definitions}
    \paragraph{Random Variable} X is a function from the sample space S to the
    real numbers $X : S \rightarrow \mathbb{R}$. Its range $R(X)$ is the set of
    possible real values it can take. We use X to denote a random variable and
    x to denote its observed value
    \begin{itemize}
      \item Discrete if takes on a countable number of possible values
      \item Continuous if it takes on all values in some interval of the real
      line
    \end{itemize}

    \paragraph{Indicator or binary} variables take values of 0 or 1

    \paragraph{Probability Function (pf)} of X, denoted $f(x)$, denotes the
    probability that X takes on the value x. $f(x) = P(X = x)$ defined for all
    x in the range of X

    \paragraph{Probability Distribution} The set of pairs $\{(x, f(x)) | x \in
    R(X))\}$ is called the probability distribution of X
    \begin{itemize}
      \item $0 \leq f(x) \leq 1$ for any $x$
      \item $\sum f(x) = 1$
    \end{itemize}

    \paragraph{Cumulaative distribution function (cdf)} of X, denoted $F(x)$
    denotes the probability that X rtakes on a value $\leq x$: $F(x) = P(X
    \leq x)$. Very useful for continuous random variables

    \paragraph{Mean or Expected Value} of X is defined as $\mu = E(X) = \sum_x
    xf(x)$. We can understand $E(X)$ as the average value that $X$ would assume
    over a theoretically infinite number of trials. $E(X)$ is not a random
    variable, it is a constant.\\
    We also define the expected value as $E[g(X)] = \sum_x g(x)f(x)$

    \paragraph{Variance} of a random variable X is the expected squared
    difference from the mean, that is $Var(X) = E[(X - E(X))^2] = \sum_{all x}
    f(x)(x-\mu)^2$. An alternative would be $Var(X) = E[X^2] - (E(X))^2$

\chapter{Discrete Probability Distributions}
  \section{Definitions}
    \paragraph{Bernoulli Distribution} Repeated trials of an experiment
    \begin{itemize}
      \item Each trial can be a success or a failure
      \item The probability of a success is the same for each trial
      \item The outcomes of different trials are \textbf{independent}
      \item Let X record success or failure
    \end{itemize}
    We say that X follows a \textbf{Bernoulli distributionS} ($X \sim
    Bernoulli(p)$), where p is the probability of sucesss

    \[ f(x) = \left\{
      \begin{array}{l l}
        p     & \quad \text{if $x = 1$}\\
        (1-p) & \quad \text{if $x = 0$}
      \end{array}
      \right.  \]\\
      $E(X) = p$\\
      $Var(X) = E[X^2] - (E(X))^2 = p - p^2 = p(1-p)$

    \paragraph{Binomial Distribution}
      Physical setup: We perform a sequence of $n$ independent Bernoulli trials
      \begin{itemize}
        \item Each trial has two possible outcomes: success or failure
        \item Trials are independent
        \item Each trial has probability of success equal to $p$
      \end{itemize}

      $f(x) = \binom{n}{x} p^x(1-p)^{n-x}$, $x = 0, 1, 2, ... n$\\
      $E(X) = np$\\
      $Var(X) = np(1-p)$

    \paragraph{Poisson Process}
      Physical setup: Events occur randomly in time (or space) according to the
      following conditions
      \begin{itemize}
        \item Independence: The number of occurrences in disjoint (non
          overlapping) intervals are independent
        \item Individuality: Events occur singly i.e $P($two or more events
          occur simultaneously$) = 0$
        \item Homogeneity: Events occur according to a uniform (constant) rate
          or intensity ($\lambda$)
      \end{itemize}
      If events occur with an average rate of $\lambda$ per unit of time
      and X is the number of events which occur in $t$ units of time, then $X
      \sim Poisson(\lambda t)$\\
      $f(x) = \frac{e^{\lambda t}(\lambda t)^x}{x!}$, $x = 0, 1, ...$\\
      $E(X) = \lambda t$\\
      $Var(X) = \lambda t$

    \paragraph{Hypergeometric Distribution}
      Physical setup: We have a collection of N objects which can be classified
      into two distinct types, called success and failure. There are $r$ and $N
      - r$ failures. A sample of $n$ objects is selected without replacement.\\
      Let X be the number of successes selected, then X is said to follow a
      hypergeometric distribution ($X \sim Hyper(N, r, n)$). To compute the
      probability functino, note that, for X = x
      \begin{itemize}
        \item There are $\binom{N}{n}$ points in the sample space (if we do not
          consider the order of selection)
        \item There are $\binom{r}{x}$ ways too select the $x$ success objects
          from the $r$ available
        \item There are $\binom{N-r}{n-x}$ ways to select the remaining $n-x$
          failure objects from the $N-r$ available
      \end{itemize}
      $f(x) = \frac{\binom{r}{x}\binom{N-r}{n-x}}{\binom{N}{n}}$, $x = 0, 1,
      ... min(r, n)$\\
      $E(X) = \frac{nr}{N}$\\
      $Var(X) = \frac{nr(N-r)(N-n)}{N^2(N-1)}$

      \paragraph{Geometric Distribution}
        Physical Setup: Bernoulli Trials are repeated until the first success.
        Let X be the number of independent Bernoulli (p) trials until the first
        success (including the first success), then X follows a geometric
        distribution, $X \sim Geom(p)$\\
        $f(x) = p(1-p)^{x-1}$, $x = 1, 2 ...$\\
        $E(X) = \frac{1}{p}$\\
        $Var(X) = \frac{1-p}{p^2}$



\chapter{Continuous Probability Distributions}
  \section{Definitions}
    \paragraph{Continuous Random Variable} X is a function from the sample
    space to the real numbers: $X : S \rightarrow \mathbb{R}$\\
    The range $R(X)$ is continuous. Individual points in $\mathbb{R}$ must have
    a 0 probability since the interval length is 0\\

    \paragraph{Probability Density Function (pdf)} of a continuous random
    variable X, denoted $f(x)$ assigns a probability to an $x \in R(X)$. If we
    have the probability density function for X, then we can define the
    probability that X takes a value in an interval $(a, b) \subseteq R(X)$ as
    $P(A < X < b) = \int_{a}^{b} f(x)dx, (a, b) \subseteq R(X)$\\
    Properties of the probability density function:\\
    $f(x) \geq 0$, $\forall x \in R(X)$\\
    $\int_{x\in R(X)} f(x)dx = 1$

    \paragraph{Cumulative Distribution Function} of a continuous brandom
    variable X, denoted F(x), gives the probability that X takes on a value
    less than or equal to x\\
    $F(x) = P(X \leq x) = P(X < x)$\\
    Properties
    \begin{itemize}
      \item $F(-\infty) = 0$
      \item $F(\infty) = 1$
      \item $F(x)$ is non decreasing
    \end{itemize}

    \paragraph{Continuous Uniform Distribution}
      Physical setup: The probability of any subinterval of the range is
      proportional to the length of the interval. For $a < b$, if X is
      uniformly distributed on the interval $(a, b)$ then we write $X \sim U(a,
      b)$\\
      $f(x) = \frac{1}{b-a}$, $a \leq x \leq b$\\

      \[ F(x) = \left\{
        \begin{array}{l l}
          0               & \quad \text{if $x < a$}\\
          \frac{x-a}{b-a} & \quad \text{if $a \leq x \leq b$}\\
          1               & \quad \text{if $b < x$}
        \end{array}
        \right.  \]\\

    \paragraph{Mean or Expected Value} of a continuous random variable X is
    defined as\\
    $E(X) = \int_{x} xf(x) dx = \mu$\\
    Properties:
    \begin{itemize}
      \item $E[g(X)] = \int_x g(x)f(x)dx$
      \item $E[aX + bY] = aE(X) + bE(Y)$
    \end{itemize}

    \paragraph{Variance} is the expected squared difference from the mean, that
    is $Var(X) = E[(X-E(X))^2]$.\\
    If $X_1$ and $X_2$ are independent random variables, and $a, b \in
    \mathbb{R}$, then $Var[aX_1 + bX_2] = a^2 Var(X_1) + b^2Var(X_2)$

    \paragraph{Exponential Distribution}
      Physical setup: Events occur according to a Poisson process, and we
      measure the inter arrival times between events. If X is the amount of
      time until the next event in a Poisson process, then $X \sim
      Exp(\theta)$, where $\theta = \frac{1}{\lambda}$\\
      $f(x) = (\frac{1}{\theta})e^{-\frac{x}{\theta}}$, $x > 0$\\
      $F(x) = 1 - e^{-\frac{x}{\theta}}$, $x > 0$\\
      $E(X) = \theta$\\
      $Var(X) = \theta^2$

    \paragraph{Simulation}
      The most common use of the continuous uniform distribution to simulate
      other random variables.\\
      \textbf{Theorem}: If $F(x)$ is an arbitrary cdf, and $Y \sim U(0, 1)$,
      then $X = F^{-1}(Y)$ has cdf $F(x)$.\\
      We can use $Y$ to generate an observation of the random variable X:
      \begin{itemize}
        \item Generate an observation y from $Y \sim U(0, 1)$ using your
          favourite software
        \item Compute $x = F^{-1}(y)$
      \end{itemize}


\chapter{Normal Distribution}
  \section{Definitions}
    \paragraph{Normal Distribution} A continuous random variable X with range
      $(-\infty, \infty)$ has a normal distribution denoted $X \sim N(\mu,
      \sigma^2)$, if its pdf has the form $f(x) = \frac{1}{\sqrt{2\pi
      \sigma^2}}e^{-\frac{1}{2}(\frac{x-\mu}{\sigma})^2}$, $x \in \mathbb{R}$\\
      where the mean $\mu$ and variance $\sigma^2$ are parameters. \\
      $E(X) = \mu$, $Var(X) = \sigma^2$

    \paragraph{Properties}
      Let $X_1 \sim N(\mu_1, \sigma^2_1)$ and $X_2 \sim N(\mu_2, \sigma^2_2)$
      be independent\\
      $Y = aX_1 + bX_2 + c \sim N(a\mu_1 + b\mu_2 + c, a^2\sigma_1^2 +
      b^2\sigma^2_2)$\\
      If $X \sim N(\mu, \sigma^2)$, then $Z = (\frac{1}{\sigma})X -
      (\frac{\mu}{\sigma}) = \frac{X-\mu}{\sigma} \sim N(0, 1)$\\
      We also have $P(Z > z) = P(Z < -z)$ for any $z \in \mathbb{R}$

    \paragraph{Central Limit Theorem} : We use the normal distribution to
    apporximate probabilities for non normal distributions. This is possible
    because the normal distribution tends to approximate sums of randm
    variable. Although this is a Theorem about limits, we will use it when $n$
    is large, but finite to approximate the distribution of $\sum_i X_i$, or
    $\bar{X}$ by a normal distribution

    \paragraph{Independent} We say that X and Y are independent if, for all x
    and y, we have \\
    $f(x, y) = P(X = x \cap Y = y) = P(X = x)P(Y=y) = f_X(x)f_Y(y)$

    \paragraph{Central Limit Theorem --- Sum} Let $X_1$, $X_2$, $X_3$,...$X_n$
    be independent variables all having the same distribution with $E(X_i) =
    \mu$ and $Var(X_i) = \sigma^2$\\
    As $n \rightarrow \infty$, the cumulative distribution function of the
    random variable $\sum_i X_i$ approaches the cumulative distribution
    function for $N(n\mu, n\sigma^2)$\\
    The cumulative distribution function of the random variable $\frac{\sum_i
    X_i - n\mu}{\sigma \sqrt{n}}$ approaches the cumulative distribution
    function for $N(0, 1)$


    \paragraph{Central Limit Theorem --- Average} Let $X_1$, $X_2$, $X_3$,...$X_n$
    be independent variables all having the same distribution with $E(X_i) =
    \mu$ and $Var(X_i) = \sigma^2$\\
    As $n \rightarrow \infty$, the cumulative distribution function of the
    random variable $\bar{X}$ approaches the cumulative distribution
    function for $N(\mu, \frac{\mu^2}{n})$\\
    The cumulative distribution function of the random variable $\frac{\bar{X} -
    \mu} {\frac{\sigma}{\sqrt{n}}}$ approaches the cumulative distribution
    function for $N(0, 1)$

    \paragraph{Continuity Correction} can improve the approximation to a sum or
    average of discrete random variables using a normal random variable. We
    think of the center of a bar with width 1 as an integer value and the bar
    actually covering $(x - 0.5, x + 0.5)$. So instead of integrating from $(0,
    5)$ for example, we integrarte on $(-0.5, 5.5)$

\chapter{Confidence Intervals}
  \section{Definitions}
    \paragraph{Introduction to Estimation}
      Suppose that a probability distribution which serves as a model for some
      random process depends on an unknown parameter $\theta$. In order to use
      the model we have to estimate or specify a value for $\theta$ using some
      data sets collected for the random variable.\\

    \paragraph{Estimate} of a parameter $\theta$ is the value of a function of
    the observed data $y_1, y_2, ... , y_n$ and other known quantities such as
    the sample size $n$.

    \paragraph{Likelihood function} for $\theta$ is defined as $L(\theta) =
    L(\theta ; y) = P(Y = y ; \theta )$ for $\theta \in \omega$\\
    where the parameter space $\omega$ is the set of possible values for
    $\theta$\\
    \\
    Suppose that $\theta$ is the success probability in a binomial model, so
    that $Y \sim Bi(n, \theta )$. Suppose that we ran the experiment once and
    recorded $y$ successes in $n$ trials. Then \\
    $L(\theta ) = P(Y = y ; \theta ) = \binom{n}{y} \theta^y (1 - \theta) ^
    {n-y}$\\
    \begin{itemize}
      \item Hold $y$ constant and vary $\theta$.
      \item This makes $L(\theta)$ into a function of $\theta$.
      \item Then we can use calculus to choose $\theta$ to maximize $L(\theta)$
      \item This choice for $\theta$ is what we call $\hat{\theta}$, the
        maximum likelihood estimate for $\theta$.
      \item Here we would get $\hat{\theta} = \frac{y}{n}$ which agrees with
        our intuition
    \end{itemize}

    \paragraph{Maximum Likelihood Estimate} The value of $\theta$ which
    maximizes $L(\theta)$ for given data $y$ is called the maxmimum likelihood
    estimate of $\theta$. It is denoted as $\hat{\theta}$.

    \paragraph{Relative Likelihood Function} is defined as $R(\theta) =
    \frac{L(\theta)}{L(\hat{\theta}}$ for $\theta \in \omega$.\\
    Note that $0 \leq R(\theta) \leq 1$ for all $\theta \in \omega$

    \paragraph{Log Likelihood Function} is $l(\theta) = \ln L(\theta)$ for
      $\theta \in \omega$. Note that $\hat{\theta}$ maximizes $R(theta)$ and
      $l(\theta)$

    \paragraph{Estimator} We call $\tilde{\theta}$ the estimator of $\theta$
    corresponding to $\hat{\theta}$. We will always use
    \begin{itemize}
      \item $\hat{\theta}$ to denote an estimate, that is, a numerical value
      \item $\tilde{\theta}$ to denote the corresponding estimator
    \end{itemize}
    An estimator $\tilde{\theta}$ is a random variable which is a function
    $\tilde{\theta} = g(Y_1, ... Y_n)$ of the random variables $Y_1, ... ,
    Y_n$. The distribution of $\tilde{\theta}$ is called the sampling
    distribution of the estimator.\\
    List of estimators:
    \begin{itemize}
      \item $Y_i \sim Bernoulli(p)$, $\tilde{p} = \bar{Y}$
      \item $Y_i \sim Poisson(\lambda)$, $\tilde{\lambda} = \bar{Y}$
      \item $Y_i \sim Exponential(\theta)$, $\tilde{\theta} = \bar{Y}$
      \item $Y_i \sim Normal(\mu, \sigma^2)$, $\tilde{\mu} = \bar{Y}$
      \item $Y_i \sim Normal(\mu, \sigma^2)$, $\tilde{\sigma^2} = \bar{s^2}$
    \end{itemize}

    \paragraph{Unbiased} An estimator is said to be unbiased if its expected
    value equals the parameter being estimated: $E(\tilde{\theta}) = \theta$.\\
    The standard deviation of an estimator is called its standard error:
    $SE(\tilde{\theta}) = \sqrt{Var(\tilde{\theta})}$\\
    If we have two unbiased estimators for a parameter, the one with the
    smaller standard error is preferred.

    \paragraph{Interval Estimate} for $\theta$ based on observed data $y$ takes
    the form $[L(y), U(y)]$. We assume that the probability model chosen is
    correct and that $\theta$ is the true value of the parameter.

    \paragraph{Coverage Probability} To quantify the uncertainty in the
    interval estimate, we define the coverage probability for the interval
    estimator $[L(Y), U(Y)]$ as \\
    $C(\theta) = P(L(Y) \leq \theta \leq U(Y))$

    \paragraph{Confidence Intervals} A 100p\% confidence interval for a
    parameter $\theta$ is an interval estimate $[L(y), U(y)]$ for which\\
    $P(L(Y) \leq \theta \leq U(Y)) = p$. We say that we are 100p\% confident
    that the true parameter is in the interval

    \paragraph{Pivotal Quantity} is a function of the data Y and the unknown
    parameter $\theta$ such that the distribution of the random variable Q is
    completely known.  We define it as: $Z = \frac{\bar{X} -
    \mu}{\frac{\sigma}{\sqrt{n}}} \sim N(0, 1)$. \\
    To compute a 100p\% confidence interval for $\mu$, determine the threshold
    $c$ such that $P(Z \leq c) = 1 - (\frac{1-p}{2}) = \frac{p+1}{2}$, where $Z
    \sim N(0, 1)$.\\
    We then construct the interval $\bar{x} \pm c\frac{\sigma}{\sqrt{n}}$

\chapter{Confidence Interval II}
  \section{Definitions}
    \subsection{Words}
      \paragraph{Treatment} procedure, machine or process being compared by our experiment
      \paragraph{Unit} object being studied by our experiment
      \paragraph{Response variable} a variable of interest whose value we measure during our experiment
      \paragraph{Explanatory Variable} a variable which influences the value of the response variate

    \subsection{Methods}
      \paragraph{Independent Sampling} Independent samples are those samples selected from the same population, or different populations, which have no effect on one another. That is, no correlation exists between the samples.
      \paragraph{Matched Pair Samples} Matched samples can arise in the following situations:\\

        Two samples in which the members are clearly paired, or are matched explicitly by the researcher. For example, IQ measurements on pairs of identical twins. \\

        Those samples in which the same attribute, or variable, is measured twice on each subject, under different circumstances. Commonly called repeated measures. Examples include the times of a group of athletes for 1500m before and after a week of special training; or the milk yields of cows before and after being fed a particular diet.\\

        Sometimes, the difference in the value of the measurement of interest for each matched pair is calculated, for example, the difference between before and after measurements, and these figures then form a single sample for an appropriate statistical analysis.

  \section{Confidence Intervals}
    \subsection{Difference of Means for large samples}
      Suppose $X_i$ and $Y_j$ are independent normally distributed samples of size $n_X$ and $n_Y$ respectively:
      \begin{center}
        $X_i \sim N(\mu_X, \sigma^2_X)$
      \end{center}
      \begin{center}
        $X_i \sim N(\mu_X, \sigma^2_X)$
      \end{center}
      For large values of $n_X$ and $n_Y$, by CLT, the random variable:
      \begin{center}
        $\frac{(\bar{X} - \bar{Y}) - (\mu_X - \mu_Y)}{\sqrt{\frac{\sigma_X^2}{n_X} + \frac{\sigma^2_Y}{n_Y}}} \sim N(0, 1)$
      \end{center}

      An approximate $100p\%$ confidence interval for $\mu_x - \mu_y$ is given by
      \begin{center}
        $ (\bar{x} - \bar{y}) \pm c \sqrt{ \frac{s^2_X}{n_X} + \frac{s^2_Y}{n_Y} } $
      \end{center}
      where $s_X$ and $s_Y$ are the \textbf{sample} standard deviations. For $Z \sim N(0, 1)$, we choose $c$ to satisfy
      \begin{center}$P(-c \le Z \le c) = p$\end{center}

    \subsection{Difference of Means for small samples}
      Keep the basic setup from earlier, except now with smaller sample sizes. Suppose that $X_i$ and $Y_i$ are independent samples of sizes $n_X$ and $n_Y$ respectively. For \textbf{small} values of $n_X$ and $n_Y$, the random variable
      $\frac{(\bar{X} - \bar{Y}) - (\mu_X - \mu_Y)}{S_p\sqrt{\frac{1}{n_X} + \frac{1}{n_Y}}} \sim t_{n_X + n_Y - 2}$
      Explanation for $S^2_p$:
      \begin{center}$S^2_p = \frac{(n_X - 1) S^2_X + (n_Y - 1) S^2_Y}{n_X + n_Y - 2}$\end{center}
      \begin{center}$S^2_X = \frac{\sum_{i=i}^{n_X} (X_i - \bar{X})^2}{n_X - 1}$\end{center}
      \begin{center}$S^2_Y = \frac{\sum_{i=i}^{n_Y} (Y_i - \bar{Y})^2}{n_Y - 1}$\end{center}

        We use the unbiased estimater $S^2_p$ for $sigma^2$ rather than the maximum likelihood estimater $\widetilde{\sigma}$

        An approximate $100p\%$ confidence interval for $\mu_x - \mu_y$ is given by
        \begin{center}$(\bar{x} - \bar{y}) \pm cs_p\sqrt{\frac{1}{n_X} + \frac{1}{n_Y}}$\end{center}
          where $c$ is chosen to satisfy $P(-c \leq t_{n_X + n_Y - 2} \leq c) = p$

    \subsection{Difference of means, not independent}
    Suppose we have paired observations $(x_i, y_i)$, where $X_i \sim N(\mu_X, \sigma^2)$ and $Y_i \sim N(\mu_Y, \sigma^2)$ are two random variables measured from the same unit of the population\\

    Define new variables $D_i = X_i - Y_i \sim N(\mu_D, \sigma^2_D)$ with observations
    \begin{center}$d_i = x_i - y_i$\end{center}

    We have the pivotal quantity $\frac{\bar{D} - \mu_D}{\frac{S_D}{\sqrt{n}}} \sim t_{n-1}$
    where $P(-c \leq t_{n-1} \leq c) = p$ and $s^2_D = \frac{\sum_{i=1}^{n}(d_i - \bar{d})^2}{n-1}$\\

    The $100p\%$ confidence interval for $\mu_X - \mu_Y$ is
    \begin{center}$(\bar{x} - \bar{y}) \pm c\frac{s_D}{\sqrt{n}}$\end{center}
      where $P(-c \leq T \leq c) = p$ and $T \sim t_{n-1}$

    \subsection{Exact Confidence Intervals}
      $X_i \sim N(\mu, \sigma^2)$\\
      A $100p\%$ confidence for $\mu$ where $\sigma$ is unknown) is given by
      \begin{center}$\bar{x} \pm c\frac{s}{\sqrt{n}}$\end{center}
      where $P(-c \leq T \leq c) = p$ and $T \sim t_{n-1}$\\

      A $100p\%$ confidence interval for $\sigma$ is given by
      \begin{center}$[ \sqrt{\frac{(n-1)s^2}{b}}, \sqrt{\frac{(n-1)s^2}{a}} ]$\end{center}
        where $P(a \leq X \leq b) = p$ and $X \sim x^2_{n-1}$

    \subsection{Approximate Confidence Intervals}
      An approximate $100p\%$ confidence interval for $\theta$ (Binomial success probability) is given by
      \begin{center}$\hat{\theta} \pm c\sqrt{\frac{\hat{\theta} (1 - \hat{\theta})}{n}}$\end{center}
      where $P(-c \leq Z \leq c) = p$ and $Z \sim N(0, 1)$

\chapter{Hypothesis Testing}
  \section{Steps}
    \begin{itemize}
      \item Define a null hypothesis and alternative hypothesis
      \item Define a test statistic which is used as evidence against the null hypothesis
      \item Calculate the $p-$ value from the test statistic, with a given threshold($\alpha$)
      \item If the $p-$value is less than the threshold, then we reject the null hypothesis
      \item Always state conclusions in the language of the original problem
    \end{itemize}

  \section{Difference between One sided and Two sided}
    For a null hypothesis, $H_0: \mu = \mu_0$, we consider two types of alternative hypothesis
    \begin{itemize}
      \item One sided: $H_a : \mu > \mu_0$ or $H_a : \mu < \mu_0$
      \item Two sided: $H_a : \mu \neq \mu_0$
    \end{itemize}
    Which altenrative hypothesis is appropriate depends on the context of the problem

  \section{Formulas}
    For $\mu \neq \mu_0$:
    \begin{center}$ |T| = \frac{|\bar{D}|}{\frac{S_D}{\sqrt{n}}} \sim |t_{n-1}| $\end{center}
    \begin{center}$ |Z| \sim |N(0, 1)| $\end{center}

    For $\mu > \mu_0$:
    \begin{center}$ T = \frac{\bar{D}}{\frac{S_D}{\sqrt{n}}} \sim t_{n-1} $\end{center}
    \begin{center}$ Z \sim N(0, 1) $\end{center}

    For $\mu < \mu_0$:
    \begin{center}$ -T = -\frac{\bar{D}}{\frac{S_D}{\sqrt{n}}} \sim -t_{n-1} $\end{center}
    \begin{center}$ -Z \sim -N(0, 1) $\end{center}


  \section{Contingency tables}
    \begin{itemize}
      \item Contingency tables contain the observed counts between two random variables
      \item Test $H_0$ : The variables are independent
      \item If $r_i$ and $c_j$ are the row and column totals, then the expected counts are
        \begin{center}$e_{ij} = \frac{r_ic_j}{n}$\end{center}
      \item Test statistic: Under $H_0$, when $n$ is large, the pivotal quantity
        \begin{center}$\bigwedge = 2 \sum_{i=1}^a \sum_{j=1}^{b} Y_{ij} \ln{(\frac{Y_{ij}}{E_{ij}})}$\end{center}
          has a $x^2_{(a-1)(b-1)}$ distribution
    \end{itemize}

\end{document}
