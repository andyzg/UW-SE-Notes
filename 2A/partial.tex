\documentclass[12pt]{report}
\usepackage{dcolumn}
\usepackage{listings}
\newcolumntype{d}[1]{D{.}{\cdot}{#1} }
\usepackage[utf8]{inputenc}
\usepackage[margin=2cm]{geometry}
\usepackage{amsfonts}
\usepackage{amsmath}
\usepackage{amssymb}
\usepackage[hidelinks]{hyperref}

\title{ECE222}
\author{{Andy Zhang}}
\date{{Fall 2014}}
\begin{document}
\chapter{Processors - September 11}
    % TODO: Get some class notes
  \section{Processor}
    \paragraph{PC} program counter stores memory address of next instruction
    \paragraph{IR} instruction register stores instruction read from memory
    \paragraph{MAR} memory adderss to register outputs address to memory
    \paragraph{MDR} memory data register. Holds data/instructions from memory
    or going to memory
  \section{Instruction execution}
    \subsection{Instruction Fetch(IF)}
      \begin{itemize}
        \item Copy PC contents to MAR and assert R/W control signal
        \item Wait for response from memory and copy MDR contents to IR
        \item Increment PC
      \end{itemize}
    \subsection{Instruction Decode(ID)}
      \begin{itemize}
        \item Interpret bits in IR
      \end{itemize}
    \subsection{Operand Fetch(OF)}
      \begin{itemize}
        \item Read data from registers and/or extract constants from IR
      \end{itemize}
    \subsection{Execute(EX)}
      \begin{itemize}
        \item Use ALU or read memory(load) or write memory(store)
      \end{itemize}
    \subsection{Writeback(WB)}
      \paragraph{} Write result to a register
      \paragraph{Eg} Execute Load R2, LOC (memory address label)
      \begin{enumerate}
        \item Always same as above
        \item Recognize "Load"
        \item Etract LOC from IR
        \item Copy LOC to MAR and assert R/W control signal
        \item Copy MDR Contents to R2
      \end{enumerate}

    \subsection{Homework}
      \begin{lstlisting}
        ADD R4, R2, R3 ($R4 <- [R2] + [R3]$)
        Store R4, LOC
      \end{lstlisting}

  \section{Design Paradigms}
    \subsection{CISC} Complex Instruction Set Computer\\
      \begin{itemize}
        \item Machine instructions can perform complex operations
        \paragraph{E.g.} (x86) \textit{movsb} copies an array of bytes
        \item Instructions are variable length
        \item Operands come from registers or memory
        \paragraph{E.g} \textit{M68K} ADD DO, LOC (mem[LOC] \textless- [DO] +
        [mem[LOC])
        \item Complex addressing modes
        \paragraph{E.g.} (M68K) ADD DO, (A0)+
        \item Smaller object code
        \item Direct support of High Level Language constructs
        \item Ease of assembly language programming
        \item Hardware is difficult to pipeline(speed up)
      \end{itemize}

    \subsection{RISC}
      Reduced instruction-set computer
      \begin{itemize}
        \item Fewer, simpler instructions
        \item Load/store architecture
        \begin{itemize}
          \item only load or store
          \item ALU operands only come from registers
        \end{itemize}
      \paragraph{Eg} (ARM)
      \begin{lstlisting}
        ldr r1, LOC
        add r1, r0, r1
        ldr r2=LOC
        str r1, [r2]
      \end{lstlisting}
      \item Object code is larger (by ~\%30)
      \item Hardwire easier to pipeline
      \end{itemize}

  \section{Register Transfer Notation}
    (no standard)
    \begin{itemize}
      \item Expresses the semantics of instruction execution as data transfers
      and control flow(logic)
      \item Memory locations are assigned labels e.g. LOC, A
      \item Registers are named R0, R1, PC, IR
      \item [$x$] denotes contents of $x$
    \end{itemize}
    \paragraph{E.g.}
    \begin{itemize}
      \item [[LOC]] contents of memory at LOC
      \item [[R0]] contents of register R0
      \item [[[R0]]] contents of memory at the location specified by contents of R0
    \end{itemize}
    `,' denotes parallel\\
    `;' dnotes sequential\\

    \paragraph{E.g.}
      \begin{lstlisting}
        ADD R4, R2, R3
        R4 <- [R2] + [R3]
      \end{lstlisting}

    \paragraph{E.g.} instruction fetch
      \begin{lstlisting}
        MAR <- [PC], R/W <- 1, PC <- [PC] + 4
        IR <= [MOR]
      \end{lstlisting}


\end{document}
