\documentclass[12pt]{report}
\usepackage{dcolumn}
\usepackage{listings}
\newcolumntype{d}[1]{D{.}{\cdot}{#1} }
\usepackage[utf8]{inputenc}
\usepackage[margin=2cm]{geometry}
\usepackage{amsfonts}
\usepackage{amsmath}
\usepackage{amssymb}
\usepackage{enumitem}
\setlist{nolistsep}
\usepackage[hidelinks]{hyperref}

\title{ECE222}
\author{{Andy Zhang}}
\date{{Fall 2014}}
\begin{document}
\maketitle
\tableofcontents
\chapter{Computers}
  \section{Classes (1.1)}

    \subsection{Personal}
      \begin{itemize}
        \item Desktop, laptop, tablet, smartphones
        \item ~\%1 of all CPUs sold(10 billion in 2008)
        \item Cost: \$20 - 200
      \end{itemize}
    \subsection{Embedded}
      \begin{itemize}
        \item Integrated into a larger device or system
        \begin{itemize}
          \item Automotive(airbags, ABS, ... )
          \item Appliances: stove, microwave...
          \item Airplanes
        \end{itemize}
        \item ~99\% of all CPUs
        \item Cost: Microchip PIC12: \$0.41
      \end{itemize}

    \subsection{Servers}
      \begin{itemize}
        \item Provides service to many users
        \begin{itemize}
          \item Cloud computing (Amazon EC2, Azure ...)
          \item Mainframes (IBM System Z) used by banks, universities,
          governments due to high reliability
          \item Supercomputers --- weather modelling, protein folding..
        \end{itemize}
        \item \textless 1\% of all CPUs sold
        \item cost: ~\$2000 / chip
      \end{itemize}

  \section{Structure (1.2)}
    \paragraph{Definition}: a computer is a `programmable device that can
    store, retrieve and process data' \\ --- Merriam Webster
    \\
    \\
    Computers of all classes can be decomposed into five types of functional
    events
    \begin{enumerate}
      \item Input: Mouse, Punchard, Touch Screen, Camera
      \item Output: Printer, Screems.
      \item Storage: Data, instructions (binary)
      \begin{itemize}
        \item Memory is organized into a linear array of bytes
      \end{itemize}
      \item ALU: Arithmetic Logic Unit
      \begin{itemize}
        \item Performs operations on data stored in registers
        \item Add, multiply, AND, NOT, ...
      \end{itemize}
      \item Control Unit
      \begin{itemize}
        \item Interpret instructions, fetch operands, control ALU
      \end{itemize}
    \end{enumerate}

\chapter{Processors}
  \section{Processor}
    \paragraph{PC} program counter stores memory address of next instruction
    \paragraph{IR} instruction register stores instruction read from memory
    \paragraph{MAR} memory address to register outputs address to memory
    \paragraph{MDR} memory data register. Holds data/instructions from memory
    or going to memory

  \section{Instruction execution}
    \subsection{Instruction Fetch (IF)}
      \begin{itemize}
        \item Copy PC contents to MAR and assert R/$\bar{W}$ control signal
        \item Wait for response from memory and copy MDR contents to IR
        \item Increment PC
      \end{itemize}

    \subsection{Instruction Decode (ID)}
      \begin{itemize}
        \item Interpret bits in IR
      \end{itemize}

    \subsection{Operand Fetch (OF)}
      \begin{itemize}
        \item Read data from registers and/or extract constants from IR
      \end{itemize}

    \subsection{Execute (EX)}
      \begin{itemize}
        \item Use ALU or read memory (load) or write memory (store)
      \end{itemize}

    \subsection{Writeback (WB)}
      \paragraph{} Write result to a register
      \paragraph{Eg} Execute Load R2, LOC (memory address label)
      \begin{enumerate}
        \item Always same as above
        \item Recognize `Load'
        \item Extract LOC from IR
        \item Copy LOC to MAR and assert R/$\bar{W}$ control signal
        \item Copy MDR Contents to R2
      \end{enumerate}

    \subsection{Homework}
      \begin{lstlisting}
        ADD R4, R2, R3 (R4 <- [R2] + [R3])
        Store R4, LOC
      \end{lstlisting}

  \section{Design Paradigms}
    \subsection{CISC} Complex Instruction Set Computer\\
      \begin{itemize}
        \item Machine instructions can perform complex operations
        \paragraph{E.g.} (x86) \textit{movsb} copies an array of bytes
        \item Instructions are variable length
        \item Operands come from registers or memory
        \paragraph{E.g} \textit{M68K} ADD DO, LOC (mem[LOC] \textless- [DO] +
        [mem[LOC]])
        \item Complex addressing modes
        \paragraph{E.g.} (M68K) ADD DO, (A0)+
        \item Smaller object code
        \item Direct support of High Level Language constructs
        \item Ease of assembly language programming
        \item Hardware is difficult to pipeline (speed up)
      \end{itemize}

    \subsection{RISC}
      Reduced instruction-set computer
      \begin{itemize}
        \item Fewer, simpler instructions
        \item Load/store architecture
        \begin{itemize}
          \item only load or store
          \item ALU operands only come from registers
        \end{itemize}
      \paragraph{Eg} (ARM)
      \begin{lstlisting}
        ldr r1, LOC
        add r1, r0, r1
        ldr r2=LOC
        str r1, [r2]
      \end{lstlisting}
      \item Object code is larger (by $\sim$\%30)
      \item Hardwire easier to pipeline
      \end{itemize}

  \section{Register Transfer Notation}
    (no standard)
    \begin{itemize}
      \item Expresses the semantics of instruction execution as data transfers
      and control flow (logic)
      \item Memory locations are assigned labels e.g. LOC, A
      \item Registers are named R0, R1, PC, IR
      \item [$x$] denotes contents of $x$
    \end{itemize}
    \paragraph{E.g.}
    \begin{itemize}
      \item \lbrack{}LOC\rbrack{} contents of memory at LOC
      \item \lbrack{}R0\rbrack{} contents of register R0
      \item \lbrack\lbrack{}R0\rbrack\rbrack{} contents of memory at the location specified by contents of
        R0
    \end{itemize}
    `,' denotes parallel\\
    `;' dnotes sequential

    \paragraph{E.g.} ADD R4, R2, R3
      \begin{lstlisting}
        R4 <- [R2] + [R3]
      \end{lstlisting}

    \paragraph{E.g.} instruction fetch
      \begin{lstlisting}
        MAR <- [PC], R/$\bar{W}$ <- 1, PC <- [PC] + 4
        IR <= [MOR]
      \end{lstlisting}

  \section{Memory}
    \begin{itemize}
      \item A processor can access a finite amount of physical memory,
      determined by the \# of address pins
      \item Memory is measured in binary units but reported with S.I. prefixes
      (e.g. 1kB = 1024 bytes)
      \item Hard disks are measured in decimal units (e.g. 1kB = 1000 bytes)
      \item IEC introduced binary units to eliminate confusion (e.g. 1kiB = 1024
      bytes)
    \end{itemize}
    \paragraph{Endianness}
    \begin{itemize}
      \item Big endian: MSB (most sig.\ byte) is at low address, LSB at high
      address
      \item Little-endian: MSB at high address, LSB at low address
    \end{itemize}
    \paragraph{Ex:} \textbf{12}34AB\textbf{CO}
    \begin{center}
      \begin{tabular}{c c}
        Big Endian &  \\
        0 & 12 \\
        1 & 34 \\
        2 & AB \\
        3 & CO \\
      \end{tabular}
      \begin{tabular}{c c}
        Little Endian & \\
        0 & CO \\
        1 & AB \\
        2 & 34 \\
        3 & 12 \\
      \end{tabular}
    \end{center}
\chapter{ARM}
  \section{Background}
    \begin{itemize}
      \item Acon/Adoned RISC Machines
      \item License designs to other companies to manufacture
      \item Target low power/low cost
    \end{itemize}
    Documentation:
    \url{http://infocenter.arm.com/help/index.jsp}

  \section{Design Principles}
    RISC but with some CISC characteristics\\
    RISC\:
    \begin{itemize}
      \item Fixed instruction size
      \item load/store architecture
    \end{itemize}
    CISC\:
    \begin{itemize}
      \item Autoincrement/decrement addressing modes
      \item Move multiple values from registers to memory, in 1 instruction
      \item Condition codes
    \end{itemize}
  \section{Memory}
    \begin{itemize}
      \item Data sizes:
        \begin{itemize}
          \item Word = 32 bits
          \item Half word = 16 bits
          \item Byte = 8 bits
        \end{itemize}
      \item Word addresses are `word aligned' (multiple of 4)
      \item Little or big endian
      \item Loads of half words or bytes at  200 extended or sign extended to
        32 bits
    \end{itemize}

  \section{Registers}
    Registers\:
    \begin{itemize}
      \item All registers are 32 bits
      \item 13 General purpose registers\: R0-R12, and
        \begin{itemize}
          \item[R13] is the stack pointer (SP)
          \item[R14] is the link register (LR)
          \item[R15] is the program counter (PC)
        \end{itemize}
      \item Condition code flags\:
        \begin{itemize}
        \item[R28 (V)] Overflow
        \item[R29 (C)] Carry out
        \item[R30 (Z)] Zero
        \item[R31 (N)] Negative
        \end{itemize}
      \item Program status register
    \end{itemize}

  \section{Instruction Set}
    \subsection{Variations}
    3 variations
    \begin{itemize}
      \item ARM --- 32 bit
        Thumb --- 16 bit (compact, limited instructions, 8 operands)
        Thumb 2 --- Mix of 16 and 32 bit, Cortex M3 in lab
    \end{itemize}

    \subsection{Data Processing Instructions}
    Most have this format\\
    \begin{lstlisting}
      <op>{flags}{cond} Rd, Rn, Op2
    \end{lstlisting}
    \begin{itemize}
      \item[Op2] Operand 2 (right operand)
      \item[Rn] Source register (Left operand)
      \item[Rd] Destination register
      \item[\{cond\}] Execute if condition is true
      \item[\{flags\}] E.g.\ S =\textgreater{} set condition code flags
      \item[\textless{} op\textgreater] Operation meumonic
    \end{itemize}
    \paragraph{E.g.}
    \begin{lstlisting}
      ADDEQ R2, R0, #1
      if Z==1, R2 <- [R0] + 1
    \end{lstlisting}

    Operand 2
    \begin{itemize}
      \item an 8 bit constant (optionally rotated)
      \item Register value (Rm) optionally shifted
      \begin{itemize}
        \item[LSL] Logical Shift Left, shift all bits to the left with 0 as LSB
          (multiply)
        \item[LSR] Logical Shift Right, shift all bits to the right with 0 as
          MSB (unsigned divide by two)
        \item[ASR] Arithmetic Shfit Right, shift all bits to the right, MSB
          becomes itself (signed divide by two)
        \item[RDR] Rotate right
        \item[RRX] Rotate right extend
      \end{itemize}
    \end{itemize}
    \paragraph{E.g.}
    \begin{lstlisting}
      ADD r2, r0, r1, LSL #2
      r2 <- [r0] + [r1] << 2
      = r2 <- [r0] + 4*[r2]
    \end{lstlisting}

    \begin{itemize}
      \item[Arithmetic] ADD, ADC (add with carry), SUB, SBC, RSB (reverse
      subtract)
      \item[Logical] (bitwise) AND, ORR, EOR, BIC (and not, bitwise clear), ORN
    \end{itemize}
    \begin{lstlisting}
      (or not), (no NOT) --- EOR Rd, Rn \#0xffffff
    \end{lstlisting}


    \begin{lstlisting}
      ADD R2, R0, #1
      ADD R2, R0, R1, LSL, #2
    \end{lstlisting}
    \begin{itemize}
      \item $R_d => R2$
      \item $R_n => R0$
      \item $op2 => \#1$ and $R1, LSL, \#2$
    \end{itemize}

  \section{D.P. Instructions}
    \subsection{Data movement}
      \begin{lstlisting}
        MOV{S}{cond} Rd, Op2
        MOV{cond}    Rd, #imm16 <---- zero extended to 32 bits
        MVN{S}{cond} Rd, Op2   ; move not
        MOVT{cond}   Rd, #imm16
      \end{lstlisting}
      \paragraph{E.g.} move $0xabcd1234$ into $R0$
      \begin{lstlisting}
        MOV R0, #0x1234
        MOVT R0, #0xabcd
      \end{lstlisting}
      \paragraph{E.g.}
      \begin{lstlisting}
        MVN R0, #1
        R0 <--- NOT 1
        = R0 <--- 0xfffffffe
      \end{lstlisting}

    \subsection{Shift}
      \begin{lstlisting}
        format. <op>{S}{cond} Rd, Rm, <Rs|#n>
        <op>: ASR, LSL, LSR, RGR, RRX
      \end{lstlisting}
      \paragraph{E.g.}
      \begin{lstlisting}
        LSL R1, R0, R2 <---- contain n
        // R1 <- [R0] << n
      \end{lstlisting}

  \section{Memory Access Instructions}
    \paragraph{Format}
    \begin{lstlisting}
      <op>{size}{cond} Rd, <address>
    \end{lstlisting}
    \begin{itemize}
      \item[address] Register indirect addressing
      \item[Rd] Destination reg (loads), source reg (stores)
      \item[cond] Only execute if true
      \item[size] B = byte, H = halfword, SB = Signed byte, SH = Signed half
      word (sign extended to 32 bits on load). Size defaults to word
      \item[op] LDR (load register) or STR (store register)
    \end{itemize}
    With immediate offset:
    \begin{lstlisting}
      <address> = [Rn {, #offsets}]
    \end{lstlisting}
    \paragraph{E.g.}
    \begin{lstlisting}
      LDR R1, [R0]
      // r1 <- [[R0] + 0]
    \end{lstlisting}
    With register offset
    \begin{lstlisting}
      <address> = [Rn, Rm{, LSL #n}]
    \end{lstlisting}
    \paragraph{E.g.}
    \begin{lstlisting}
      LDR R1, [R0, -R2]
      // R1 <- [[R0] - [R2]]
      LDR R1, [R0, R2, LSL #2]
      // R1 <- [[R0] + [R2] << 2]
    \end{lstlisting}

  \subsection{Sum Array}
    See the demo0.s sheet.


\section{Memory}
  \subsection{Definitions}
    \paragraph{DCD} = Declare Data
    \begin{lstlisting}
      DCD 5
      DCB 1, 2, 3, 4, 5
    \end{lstlisting}
    This is very similar to declaring
    \begin{lstlisting}
      const int N = 5
      const int ARRAY[] = [1, 2, 3, 4, 5]
    \end{lstlisting}

    \paragraph{ADR} = Put address in register
    \begin{lstlisting}
      int *ptr = ARRAY,
      int total = ptr[0];
      int a = ptr[1];
      total += a;
    \end{lstlisting}

  \paragraph{Example} ~\\
    pc relative: LDR\{size\}\{cond\} Rt, label
    \begin{itemize}
      \item label is translated into an offset from the instruction
      \item data is loaded from \textless address\textgreater = PC + 4 + offset
      \item offset $\in$ [[ pc, \#28 ]]
    \end{itemize}

  \paragraph{E.g.}
  \begin{lstlisting}
    LDR r1, DATA1 ; DATA 1 is 322 bytes after LDR
    =>
    LDR r1, [pc, #28] ; pc incremented by 4 during fetch
    // r1 <- [[pc] + 28]
  \end{lstlisting}

  load address into register: ADR\{cond\} Rd, label
  \paragraph{e.g.}
  \begin{lstlisting}
    ADR r1, DATA1
    // r1 <- [pc] + 28
  \end{lstlisting}

\section{Pseudo Instructions}
  \begin{itemize}
    \item don't match an existing machine instruction
    \item translated by the assembler into an appropriate instruction(s)
    \paragraph{e.g.}
    \begin{lstlisting}
      LDR {cond} Rt = <expr> ; label or numeric expression at end
    \end{lstlisting}
    \item converted into
    \begin{itemize}
      \item MOV or MVN, if possible, or
      \item adds the value to the literal pool (program contants, bottom of
      file) and generates a LDR instruction with pc + relative addressing from
      demo0.s file
      \begin{lstlisting}
        LDR r1 = SUM
      \end{lstlisting}
      \item couldn't use ADR because the distance from LDR (0x114) to sum (0x1000
      0000) exceeds $\pm 4095$
    \end{itemize}
    \item Stores 0x1000 0000 at address 0x134 and replaces it with:
    \begin{lstlisting}
      LDR r1, [pc, #28]
    \end{lstlisting}
  \end{itemize}

\section{Branch Instruction}
  \begin{itemize}
    \item Changes control flow by adding an offset to the PC
    \item Format: B\{cond\} label
    \begin{itemize}
      \item[cond] only execute if condition is true
      \item[label] assembler replaces with pc relative offset
    \end{itemize}
    \item condition code suffixes (SIGNED)
    \begin{itemize}
      \item[EQ] equal (to zero)
      \item[NE] not equal
      \item[GT] greater than
      \item[GE] greater or equal
      \item[LT] less than
      \item[LE] less or equal
      \item[PL] plus ($>= 0$) ignores overflow
      \item[MI] minus ($<0$) ignores overflow
    \end{itemize}
    \item condition code suffixes (UNSIGNED)
    \begin{itemize}
      \item[VS] overflow set
      \item[VC] overflow clear
      \item[Al] always (default)
    \end{itemize}

    \paragraph{E.g.} SUBS r2, r2, \#1 // r2 <- [r2] - 1
    \begin{lstlisting}
      BGT LOOP
      do { int a = *ptr;
           total += a;
           ptr++;
           counter--;
         } while (counter > 0);} // demoX.s
    \end{lstlisting}
  \end{itemize}


\section{ARM}
  \subsection{Pre and Post indexed Addressing}
    \begin{itemize}
      \item applies to LDR and STR
      \item pre indexed: \textless op\textgreater \{size\}\{cond\} Rt, [[Rn,
      \#offset]]
      \begin{itemize}
        \item the offset is added to the address in Rn, then the memory acess
        is performed and Rn is updated
      \end{itemize}

      \paragraph{E.g.}
      \begin{lstlisting}
        LDR r1, [r0, #4]!
        // r1 <- [[r0] + 4], r0 <- [r0] + 4
      \end{lstlisting}

      \item post indexed: \textless op\textgreater \{size\}\{cond\} Rt, [[Rn]],
      \#offset
      \begin{itemize}
        \item the memory access is performed with the address in Rn, then Rn is
        updated by adding the offset
      \end{itemize}
    \end{itemize}

    \paragraph{E.g.}
    \begin{lstlisting}
      LDR r1, [r0], #4
      // r1 <- [[r0]], r0 <- [r0] + 4
    \end{lstlisting}

\section{Compare Instructions}
  \begin{itemize}
    \item Compares two operands and sets the condition flags(N, Z, C, V) but does not
    save to a destination register\\
    \item CMP \{cond\} Rn, Operand2. Example:
    \begin{lstlisting}
      CMP r1, #1
      // N, Z, C, V <= [r1] - 1
      if r1 = 1, N <- O, Z <- 1, C <- 1, V <- 0
        but r1 is not changed
    \end{lstlisting}

    \item CMN\{cond\} Rn, Operand2 // compare negative
    \begin{itemize}
      \item compares [[Rn]] and ---Operand2
    \end{itemize}
    \item test: TST\{cond\} Rn, Operand2
    \begin{itemize}
      \item perform bitwise AND and updates flags N, Z, C, V. E.g. :
      \begin{lstlisting}
        TST r1, \#0x00008000
        // N, Z, C, V <= [r1] * 2_0000 0000 0000 0000 1000 0000 0000 0000
        if r1 = 0xF000 -> 0000 0000 0000 0000 1111 0000 0000 0000
                          0000 0000 0000 0000 1000 0000 0000 0000
      \end{lstlisting}
    \end{itemize}

    \item test equal: TEQ\{cond\} Rn, Operand2
    \begin{itemize}
      \item performs bitwise EOR and updates flags
    \end{itemize}
    \item compare and branch: \textless op\textgreater Rn, label. For op:
    \begin{itemize}
      \item[CBZ] compare equal to zero
      \item[CBNZ] compare not equal to zero
    \end{itemize}
    \item compares [[Rn]] with zero and decides on branch
    \begin{itemize}
      \item[CBZ] Rn, label =
      \begin{itemize}
        \item[CMP] Rn, \#0
        \item[BEQ] label
      \end{itemize}
      \item[CBNZ] Rn, label =
      \begin{itemize}
        \item[CMP] Rn, \#0
        \item[BNE] label
      \end{itemize}
    \end{itemize}
  \end{itemize}

\section{If Else}
  \paragraph{E.g.}
  \begin{lstlisting}
    if (x == 0) // cond
      y++; // stmt 1
    else
      y--; // stmt 2
    x in r0, y is r1
  \end{lstlisting}
  This is equivalent to
  \begin{lstlisting}
        CBZ r0, ADD_ // cond
        SUB r1, r1, #1 // stmt 2
        B END
 ADD_   ADD r1, r1, #1 // stmt 1
 END_   // end code
  \end{lstlisting}
  Without branching:
  \begin{lstlisting}
    CMP r0, #0
    ADDEQ r1, r1, #1
    SUBNE r1, r1, #1
  \end{lstlisting}

\section{Subroutines}
  \subsection{The Stack}
    \begin{itemize}
      \item LIFO
      \item Each thread/process has a call stack growing from high to low
      memory address
      \item Typical memory map (32 bit addressing)
      \item r13(alias sp) is the stack pointer
      \item Push[[r0]] onto the stack
      \begin{lstlisting}
        STR r0, [sp, #-4]! // \! == update/write, (pre-indexed)
      \end{lstlisting}
      \item Pop from stack into r0
      \begin{lstlisting}
        LDR r0, [sp], #4 // post indexed
      \end{lstlisting}
    \end{itemize}

  \subsection{Calling}
    \begin{itemize}
      \item Branch to subroutine instructions store return address in the link
      register r14 (alias lr)
      \item Branch and link: BL\{cond\} label (invocation)
      \begin{lstlisting}
        // pc <- [pc] + offset // offset in [-16MB, 16MB]
        r14 <- [return address] // pc of the next instruction
      \end{lstlisting}
      \item Branch, link and exchange: BLX {cond} Rn (invocation)
      \begin{lstlisting}
        eg: BLX r1
        // pc <- [r1], r14 <- [pc]
        // Greather range than BL [-2GB, 2GB]
      \end{lstlisting}
    \item Branch exchange: Bx {cond} Rn (return)
      \begin{lstlisting}
        e.g. Bx lr
        // pc <- [r14]
      \end{lstlisting}
    \end{itemize}

  \subsection{Load/Store Multiple}
    \begin{itemize}
      \item Push/pop multiple register values to/from stack
      \item Syntax \textless{} op\textgreater{} \{mode\}\{cond\} Rn\{!\},
      reglist
      \begin{itemize}
        \item[op] = LDM load muliple, or SDM store multiple
        \item[mode] = IA increment after, or DB decrement before
        \item[!] = writeback (update the stack pointer)
        \item[reglist] = comma separated list of regs or reg ranges
      \end{itemize}
      \item STMFD is a synonym for STMDB (push)
      \item LOMFD is a synonym for LDMIA
      \begin{itemize}
        \item[FD] = full descending stack
      \end{itemize}
      \begin{lstlisting}
        e.g.\ push r4, r5, r6 and r14 onto the stack
        STMFD sp!, [r4, r6, lr]
        r4  | 0000 0004
        r5  | 0000 0005
        46  | 0000 0006
        r13 | 1000 0200 <- sp
        r14 | 0000 0100 <- lr
        Memory after pushing
        1000 01F0 | 0000 0004 <- sp
        1000 01F4 | 0000 0005
        1000 01F8 | 0000 0006
        1000 01FC | 0000 0100
        1000 0200 | prev valu <- previous sp

        e.g.\ pop from stack into r4, r5, r6, pc
        LDMFD sp! {r4---r6, pc}
      \end{lstlisting}
    \end{itemize}

  \subsection{AAPCS (Arm Architecture Procedure Call Standard)}
    \begin{tabular}{c c c c}
      registers & synonyms & callee preserved & function\\
      r0-r3 & a1-a4   &   no  & argument/result/scratch regs\\
      r4-r1 & v1-v8   &   yes & local variable\\
      r12   & p       &   no  & intro procedure/scracthing\\
      r13   & sp      &   yes & stack pointer\\
      r14   & lr      &   no  & link register\\
      r15   & pc      &   no  & program counter
    \end{tabular}
    \paragraph{Guidelines}
    \begin{itemize}
      \item Preserve and restore v1 --- v8 (r4 --- r11) if you modify it
      \item Anything pushed in the stack must be popped/removed
      \item Return values are in r0 (and r1 --- r3 as needed)
      \item Pass parameters via registers first (faster)
      \item pass additional parameters via stack
    \end{itemize}

\end{document}
